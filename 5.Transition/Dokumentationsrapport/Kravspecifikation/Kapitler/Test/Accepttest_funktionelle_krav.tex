\section{Accepttest funktionelle krav}
Dette afsnit specificerer accepttesten af den Autonome overvågnings drone.

\subsection*{Formål}
Dokumentet specificerer accepttests og vil i udfyldt stand udgøre accepttest tilhørende funktionelle krav. 

\subsection*{Godkendelseskriterier}
Godkendelsen af accepttesten består af to trin:

\begin{enumerate}
	\item Godkendelse af accepttestspecifikationen. \\
	Dette gøres på bagsiden af dokumentet i "Godkendt af" feltet.

	\item Godkendelse af selve accepttesten. Dette gøres i afsnit Testresultat. 
\end{enumerate}

Accepttesten er afsluttet, når alle testcases er testet.

Hvis der under accepttesten opstår fejl, der umuliggør fortsat udførsel af de efterfølgende testcases afbrydes denne test.

Hvis der opstår fejl i enkelte testcases; men fortsat accepttest er mulig, underkendes den enkelte test og accepttesten fortsættes med efterfølgende testcases.

Såfremt en test afbrydes eller et testcase underkendes, skal der udfærdiges en problemrapport, der beskriver årsagen til underkendelse. Problemrapporten godkendes af både internt og af kunde eller produkt manager.

\newpage

\subsection*{Testspecifikation}
Software der skal testes:
\begin{table}[H]
	\centering
		\begin{tabular}{|c|c|c|c|}
			\hline
			Software & Version & Release dato & Bemærkning \\ \hline
			Webapplikation & xxx & &\\ \hline
			Opdater position & Rev. 302 & 20/05-2013 & \\ \hline
			Tilpas orientering & Rev. 354 & 20/05-2013 & \\ \hline
		\end{tabular}
	\caption{Software}
\end{table}

Hardware der skal testes:
\begin{table}[H]
	\centering
		\begin{tabular}{|c|c|c|c|}
			\hline
			Hardware & Version & Release dato & Bemærkning \\ \hline
			Højdemåler & Rev. 172 & 20/05-2013 &  \\ \hline
			3G modul & Rev. 172 & 20/05-2013 &  \\ \hline
			Opdater position & Rev. 296 & 20/05-2013 &  \\ \hline
			Tilpas orientering & Rev. 172 & 20/05-2013 &  \\ \hline
		\end{tabular}
	\caption{Hardware}
\end{table}


\subsection*{Testprocedure}
De individuelle use cases og scenarier i kravspecifikationen testes i enkelte test cases med testdata som angivet for test casen. 

\begin{itemize}
	\item Hvis et teststep gennemføres fejlfrit markeres dette med ”Godkendt” i feltet ”resultat” for testen. for det pågældende test step.

	\item Hvis et teststep gennemføres med ubetydelige fejl, markeres dette med ”(OK)” i feltet ”resultat” for det pågældende test step, samt evt. en henvisning til en fejlrapport hvori fejlen beskrives.
	
	\item Hvis et teststep gennemføres med betydelige fejl, markeres dette med en henvisning (”1”, ”2”, ”3”,…) til en fejlrapport, som udarbejdes med en beskrivelse af fejlen.
	
\end{itemize}
Det overordnede testresultat angives på sidste side i dette dokument. Ved kundens og projektgruppens underskrift på samme side godkendes det af begge parter, at testresultatet er som angivet.

\newpage

\subsection*{Test case 1: Start quadrocopter}
Use case under test: UC 1: Start quadrocopter.\\
Forudsætninger:	Ingen.

\textbf{Hovedscenarie}
\begin{table}[H]
	\centering
		\begin{tabular}{|l|p{5 cm}|p{5 cm}|p{3.5 cm}|} 
		\hline
			Step & Handling & Forventet resultat & Resultat\\ \hline
			1 & Bruger tænder quadrocopter. & Det indikeres at systemet er tilsluttet forsyning. &   \\ \hline
			2 & Quadrocopter intialiseres. & Arduino bootes og ESC'er signalerer de er flyve klar. &   \\ \hline
			3 & Forbindelse fra quadrocopter til webapplikation oprettes. & Fra webapplikation ses at quadrocopter er connected. &  \\ \hline
		\end{tabular}
	\caption{Test case 1: Start system}
	%\label{tab:TC1}
\end{table}


\subsection*{Test case 2: Ny flyveopsætning}
Use case under test: UC 2: Ny flyveopsætning.\\
Forudsætninger:	Bruger er oprettet i systemet og UC\#1 er succesfuld gennemført.

\textbf{Hovedscenarie}
\begin{table}[H]
	\centering
		\begin{tabular}{|l|p{5 cm}|p{5 cm}|p{3.5 cm}|} 
		\hline
			Step & Handling & Forventet resultat & Resultat\\ \hline
			1 & Bruger logger på webapplikation. & Login lykkes. &  \\ \hline
			2 & Webapplikations forside vises. & Webapplikationens forside vises. &   \\ \hline
			3 & Fra forsiden navigerer bruger til flyveopsætning. & Flyveopsætnings siden vises. & \\ \hline
			4 & Bruger laver en ny flyveopsætning. & Flyveopsætning klargjort & \\ \hline
			5 & Flyveopsætning sendes til quadrocopter. & Flyveopsætning sendes til quadrocopter & \\ \hline
		\end{tabular}
	\caption{Test case 2: Ny flyveopsætning}
	%\label{tab:TC2}
\end{table}

\textbf{Extension 1: Fejl i login.}
\begin{table}[H]
	\centering
		\begin{tabular}{|l|p{5 cm}|p{5 cm}|p{3.5 cm}|} 
		\hline
			Step & Handling & Forventet resultat & Resultat\\ \hline
			a & Bruger føres tilbage til login. & Login kan påny forsøges. & \\ \hline
		\end{tabular}
\end{table}

\textbf{Extension 2: Der laves ikke ny flyveopsætning.}
\begin{table}[H]
	\centering
		\begin{tabular}{|l|p{5 cm}|p{5 cm}|p{3.5 cm}|} 
		\hline
			Step & Handling & Forventet resultat & Resultat\\ \hline
			a & Gemt flyveopsætning bruges. & Flyveopsætning klargjort. & \\ \hline
		\end{tabular}
\end{table}

\newpage

\subsection*{Test case 3: Flyv til position}
Use case under test: UC 3: Flyv til position.\\
Forudsætninger:	UC\#1 og UC\#2 er succesfuld gennemført.

\textbf{Hovedscenarie}
\begin{table}[H]
	\centering
		\begin{tabular}{|l|p{5 cm}|p{5 cm}|p{3.5 cm}|} 
		\hline
			Step & Handling & Forventet resultat & Resultat\\ \hline
			1 & Nuværende position opdateres. & Ingen handling  &  \\ \hline
			2 & Flyvehøjde tilpasses. & Flyvehøjde justeres &  \\ \hline
			3 & Flyveorientering tilpasses. & Orientering justeres &  \\ \hline
			4 & Quadrocopter flyver mod ønsket position. & Quadrocopter nærmer sig ønsket position  &  \\ \hline
		\end{tabular}
\end{table}

\textbf{Extension 1: Ugyldig GPS koordinat.}
\begin{table}[H]
	\centering
		\begin{tabular}{|l|p{5 cm}|p{5 cm}|p{3.5 cm}|} 
		\hline
			Step & Handling & Forventet resultat & Resultat\\ \hline
			a & Quadrocopter går i fejlmode. & Forsøger at finde gyldig GPS koordinat, mislykkes dette lander quadrocopteren. & \\ \hline
		\end{tabular}
\end{table}

\textbf{Extension 2: Ugyldig flyvehøjde.}
\begin{table}[H]
	\centering
		\begin{tabular}{|l|p{5 cm}|p{5 cm}|p{3.5 cm}|} 
		\hline
			Step & Handling & Forventet resultat & Resultat\\ \hline
			a & Quadrocopter går i fejlmode. & Forsøger at finde gyldig flyvehøjde, mislykkes dette lander quadrocopteren. & \\ \hline
		\end{tabular}
\end{table}

\newpage

\subsection*{Test case 4: Billede af position}
Use case under test: UC 4: Billede af position.\\
Forudsætninger:	UC\#1, UC\#2 og UC\#3 er succesfuld gennemført.

\textbf{Hovedscenarie}
\begin{table}[H]
	\centering
		\begin{tabular}{|l|p{5 cm}|p{5 cm}|p{3.5 cm}|} 
		\hline
			Step & Handling & Forventet resultat & Resultat\\ \hline
			1 & Quadrocopter tager et billede af nuværende position. & Kamera aktiveres og der tages et billede. & \\ \hline
			2 & Billedet sendes til webapplikation. & Billedet gøres tilgængelig på webapplikationen &  \\ \hline
			3 & Bruger er online på webapplikation og giver accept af billedet. &  & \\ \hline
		\end{tabular}
	\caption{Test case 4: Tilpas højtaler retning}
	\label{tab:TC 4}
\end{table}


\textbf{Extension 1: Ugyldig positionsdataserie.}
\begin{table}[H]
	\centering
		\begin{tabular}{|l|p{5 cm}|p{5 cm}|p{3.5 cm}|} 
		\hline
			Step & Handling & Forventet resultat & Resultat\\ \hline
			a & Der måles en ugyldig positionsdataserie. & Højtaler retning indstilles til default. & Ikke godkendt – kan ikke testes 	\\ \hline
		\end{tabular}
	\caption{Test Case 4: Tilpas højtalers retning - Extension 1}
	\label{tab:TC4ex1}
\end{table}

\textbf{Extension 2: Bærbar enhed mister forbindelse til anlæg.}
\begin{table}[H]
	\centering
		\begin{tabular}{|l|p{5 cm}|p{5 cm}|p{3.5 cm}|} 
		\hline
			Step & Handling & Forventet resultat & Resultat\\ \hline
			a & Bærbar enhed mister forbindelse til anlæg. & Højtaler retning og lydstyrke indstilles til default. & Ikke godkendt – kan ikke testes\\ \hline
		\end{tabular}
	\caption{Test case 4: Tilpas højtaler retning - Extension 2}
	\label{tab:TC4ex2}
\end{table}

\newpage
\subsection*{Test case 5: Tilpas lydtryk}
Use case under test: UC 5: Tilpas lydtryk.\\
Forudsætninger:	UC 1 er succesfuldt gennemført.

\textbf{Hovedscenarie}
\begin{table}[H]
	\centering
		\begin{tabular}{|l|p{5 cm}|p{5 cm}|p{3.5 cm}|} 
		\hline
			Step & Handling & Forventet resultat & Resultat\\ \hline
			1 & Bærbar enhed måler lydtryk hos bruger. & Ingen handling. & Godkendt\\ \hline
			2 & Bærbar enhed sender data til anlæg. & Ingen handling. & Godkendt\\ \hline
			3 & Anlæg bearbejder data. & Ingen handling. & Godkendt\\ \hline
			4 & Anlæg tilpasser lydtryk. & System afspiller lyd i ønsket lydtryk \SI{+-6}{dB}. & Godkendt – Justeres dog ud fra fixed point\\ \hline
		\end{tabular}
	\caption{Test case 5: Tilpas lydtryk}
	\label{tab:TC5}
\end{table}

\textbf{Extension 1: Ugyldig lydtryksdataserie.}
\begin{table}[H]
	\centering
		\begin{tabular}{|l|p{5 cm}|p{5 cm}|p{3.5 cm}|} 
		\hline
			Step & Handling & Forventet resultat & Resultat\\ \hline
			a & Der måles en ugyldig lydtryksdataserie. & Lydstyrke indstilles til default. & Ikke godkendt  \\ \hline
		\end{tabular}
	\caption{Test Case 5: Tilpas lydtryk - Extension 1}
	\label{tab:TC5ex1}
\end{table}

\textbf{Extension 2: Bærbar enhed mister forbindelse til system.}
\begin{table}[H]
	\centering
		\begin{tabular}{|l|p{5 cm}|p{5 cm}|p{3.5 cm}|} 
		\hline
			Step & Handling & Forventet resultat & Resultat\\ \hline	
			a & Bærbar enhed mister forbindelse til system. & System går i default mode. & Ikke godkendt – kan ikke testes \\ \hline
		\end{tabular}
	\caption{Test case 5: Tilpas lydtryk - Extension 2}
	\label{tab:TC5ex2}
\end{table}




