\chapter{Foranalyse}

\section{Revisionshistorik}
\begin{table}[H]
	\centering
		\begin{tabular}{|p{2 cm}|p{2 cm}|p{3 cm}|p{6 cm}|} 
		\hline
			\textbf{Rev. Nr} & \textbf{Dato}		& \textbf{Initialer} 	& \textbf{Ændring} \\ \hline
			1.0 	& 25-08-14	& KG,RL		& Oprettet foranalyse dokument.	\\ \hline
			1.1 	& 11-09-14	& KG,RL		& Tilføjet afsnit om afstandsmåler og batteri.	\\ \hline
		\end{tabular}
	\caption{Revisionshistorik}
	%\label{tab:TC1}
\end{table}

\vspace{1.5cm}

\section{Ordforklaring}
\begin{table}[H]
	\centering
		\begin{tabular}{|p{2.5cm}|p{4.5 cm}|p{6.5 cm}|} 
		\hline
			\textbf{Forkortelse} & \textbf{Betydning} & \textbf{Forklaring} \\ \hline
			 &  &  \\ \hline
			 &  & \\ \hline
		\end{tabular}
	\caption{Ordforklaring}
	%\label{tab:TC1}
\end{table}

\vspace{2cm}

\section{Indledning}

Dette dokument indeholder foranalyse til projektet \textit{Autonom overvågnings drone}. 
Foranalysen udarbejdes for at få et overblik over projektets forskellige dele. 

I foranalysen undersøges der hvad der kræves af software og hardware. Blandt andet undersøges forskellige hardware moduler, hvordan server kan sættes op, samt udviklingsmiljøer og udviklingsværktøjer. Foranalysen som helhed bruges til at dokumentere de valg og beslutninger der dannede grund for projektets indkøbsfase. 
