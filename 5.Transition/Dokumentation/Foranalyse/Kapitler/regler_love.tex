\section{Sikkerhed og lov}

Dronen er tung og kan med sine store og hurtigt roterende propeller være både farlige og ødelæggende for sine omgivelserne. Derfor blev det besluttet at dronen under alle indledende test skulle være fastspændt. 

Udover at håndtere dronen forsvarligt, er det også nødvendigt at følge hvad den danske lovgivning siger om flyvning med droner. I dansk luftrum skal flyvning droner som vejer under 25kg ske i henhold til gældende regler, jf. BL 9-4, 3. udgave af 9. januar 2004 8. 

Nedenfor er de væsentligst regler for flyvning med droner listet op:

\begin{itemize}
	\item Flyvningen må ikke udsætte andres liv og ejendom for fare [4.1.a].
	\item Afstanden til bebyggelse og større offentlig vej skal være mindst 150 m [4.1.d].
	\item Flyvehøjden må højst være 100 m over terræn [4.1.e].
	\item Tæt bebyggede områder (sommerhusområder og campingpladser) samt områder, hvor et større antal mennesker er samlet må ikke overflyves [4.1.f].
\end{itemize}

Yderlige information om lovgivning i forbindelse med flyvning kan findes i bilag \textit{Bestemmelser bl94.}

