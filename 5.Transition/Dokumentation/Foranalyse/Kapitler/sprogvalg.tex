\subsection*{Sprogvalg}

\subsubsection*{Arduino}
Arduino er et opensource miljø som kan køre helt uafhængigt af OS. Sproget på arduino er en blanding af C og C++ alt afhængigt af hvor hardware nær koden skal være.

\subsubsection*{Server}
Omkrig server delen af projektet havde vi nogle overvejelser. Vi prioriterede højt læsbarhed, testbar og strukturering.

\subparagraph*{PHP}
PHP er et scripting sprog til serverside programmering. Lærings kurven er meget lav og sproget er meget fleksibelt. Da det er et scripting sprog er der ikke rigtig muligt at lave OOP udvikling. Med PHP udvikling ville vi gå meget på kompromis med strukturering for fleksibilitet. 

\subparagraph*{Ruby}
Ruby og Ruby On Rails er et forhåndsvist nyt udviklings værktøj/sprog til server udvikling. Lærings kurven er middel og sproget er tildels fleksibelt. Med Ruby On Rails er det muligt at skrive alt koden OOP. Sproget er tildels testbart, men det er dog ikke fokus i sproget. 

\subparagraph*{Python}
Python er et ældre programmerings sprog i modsætning til de to andre kandidater bliver python brugt til meget andet end web udvikling, derfor er test frameworket meget stort og kombineret med Django som web framework udvides dette test framework til et stort og meget stabilt test framework. Python er meget kendt den meget stramme syntaks, derfor er læringskurven stejler end hos de to andre sprog. Både django og ruby on rails er to meget agile frameworks som begge benytter sig af MVC modellen.\\

Da gruppen allerede havde lidt kendskab til python var det oplagt at vægle python+django som sprog og framework til server arbejdet. En anden fordel som python også har er det meget gode test framework.

\subparagraph*{Udvidet test framework}
I samarbejde med python og django test frameworket har gruppen også valgt at inddrage Selenium teknologien. Selenium er et værktøj som kan automatisere en browser, dvs der kan skrives kode som så kan teste GUI'et og selve funktionaliteten i webapplikationen. 

\subparagraph*{Database}
Django frameworket understøtter en række forskellige databaser. Gruppen har valgt at beskæftige sig med en SQLight database, da dette er en simpel light weight database og dækker applikations behov for at kunne gemme billeder og tekst strenge. 