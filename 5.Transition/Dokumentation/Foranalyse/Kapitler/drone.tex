\section{Drone}

Da dronen til projektet skulle vælges, blev den valgt ud fra følgende kriterier:

I dokumentet søges information og afklaring af følgende punkter:
\begin{enumerate}[label*=\arabic*.]
	\item Pris.
	\item Løftekapacitet. 
	\item Tilgængelighed af reservedele. 
	\item Open source kode til regulering (motorstyring)
\end{enumerate}

\vspace{1cm}

Ud fra kriterierne stod valget af drone mellem to forskellige quadrocopterer: 
3DR – Quadrocopter:  Det amerikanske firma 3D Robotics udvikler droner i forskellige størrelser. Quadrocopterer fra 3DR kan enten købes færdige eller som ”byg selv” projekter. Det bedst egnede til projektet var et ”byg selv” kit der hed DIY Quad Kit\footnote{https://store.3drobotics.com/products/diy-quad-kit}. 

Ved køb af kittet fås 4x 10 x 4.7 propeller, 20 Amp ESC’er (motor kontrol) og motorer der maksimalt kan rotere 850 gange pr min / volt.  Desuden medfølger et motorstyringsmodul og styringssoftwaren er open source. Den samlede pris for DIY Quad kittet ligger på 549 USD. Det bemærkes, at batteri / batterier til quadrocopteren skal købes separat.

AeroQuad: Det Amerikanske firma AeroQuad har lavet quadrocopteren AeroQuad Cyclone ARF. Aeroquad'en\footnote{http://www.aeroquadstore.com/AeroQuad\textunderscore Cyclone\textunderscore ARF\textunderscore Kit\textunderscore p/aqarf-001.htm}  

er en kraftigt quadrocopter med 12” propeller og motorer der yder 950 kv ( 950 rpm/V). Til motorstyring følger open source styringssoftware samt fire 30 amperes elektrisk hastighedsstyring med (ESC – Electric speed controller). Ydermere følger også tre pull og tre push propellere med. Ligesom det var tilfældet for 3DR byg selv kit, skal batteriet til AeroQuad også købes separat. Ønskes yderligere reservedele, kan de købes via AeruoQuad’s hjemmeside. Prisen for en AeroQaud ligger på 509\$.

Da AeroQuad’s quadrocoptor var 50 USD billigere end 3D-robotics og desuden havde mere løftekapacitet, blev det besluttet at AeroQuad quadrocopteren skulle benyttes til projektet. 

