\section{Drone}

Batteri til dronen blev valgt ud fra følgende kriterier:

\begin{enumerate}[label*=\arabic*.]
	\item Hvor stor strøm der kan trækkes fra batteri.
	\item Spændingsniveau. 
	\item Batterikapacitet – mAh
\end{enumerate}

Erfaringer baseret på tidligere studerendes bachelorprojekter viser, at batteriet belastes meget kraftigt  (30-40 ampere). Derfor påkræver projektet et batteri der momentvis kan holde til en stor belastning. Batteriets spændingsniveau skal desuden passe med det niveau quadrocopterens motorer bruger. 
Kapaciteten af batteriet skal være så stor som muligt. Stor kapacitet betyder at motorerne kan forsynes i længere tid, hvilket betyder at quadrocopteren kan blive i luften længere tid. Længere flyvetid giver bedre forudsætning til at udfører diverse test samt testflyvninger. 
Men da quadrocopteren har begrænset løftekapacitet har batteriets vægt også en betydning. Et batteri med en stor kapacitet vejer ofte mere end et batteri med en mindre kapacitet. Derfor ønskes det at finde et batteri der har stor kapacitet og samtidig ikke er for tungt.  
For at spare penge og tid på gentagende gange at skulle købe nye batterier, prioriteres det også højt at batteriet der købes er genopladeligt. 
Valget af batteri faldt på et evo 11.1 V batteri\footnote{http://hobbyfly.com/batterier-162/li-po-batterier-450/evo-111v-3s-4200mah-35-70-5c-li-po-batteri-deant-13361.html}.   Dette batteri har en typisk udgangsspænding på 11.1V, hvilket passer godt overens med motorernes maksimale forbrug på ≈10V. Batteriet tåler desuden stor belastning og er genopladeligt. Foruden batteri, blev der også købt en tilhørende lader station\footnote{http://hobbyfly.com/opladere-tilbehoer-133/hobbyfly-hbc680-digital-balance-lader-11-18v-80w-13248.html}.

