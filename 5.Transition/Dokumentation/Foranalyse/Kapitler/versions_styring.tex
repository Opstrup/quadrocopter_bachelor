\subsection*{Versions styring}
I overvejelser omkrig værktøjer til versions styring blev kontrol og frihed prioriteret højt. Specielt blev mulighed for selv at styre server og nem oprettelse af nyt repository eller redigering af gamle repository's prioriteret højt.
Det blev besluttet at bruge enten SVN eller GIT. Begge er to meget gode versions styrings værktøjer, der bidrager til god struktur og giver overblink over de dokumenter der udarbejdes og deles. 


\textbf{SVN} \\
SVN er det ældste versions styrings værktøj og er generelt det værktøj der bruges når projekter udarbejdes på Ingeniørhøjskolen. Det betyder at alle gruppens medlemmer har arbejdet med SVN og derved har kendskab til værktøjets stærke og svage sider. 
Desværre opstår der ofte problemer med SVN, da værktøjet i nogle tilfælde er dårligt til at merge filer, hvilket leder til mange merge conflicts. Disse conflicts er tidskrævende og danner grundlag for megen frustration. 

\textbf{GIT} \\
GIT er et nyere versions styrings værktøj og er derfor ikke så vel dokumenteret.
Men GIT er et meget populært værktøj, specielt inden for startup virksomheder, mindre virksomheder og open source projekter. 
En af grundene til GIT er et populært værktøj er, at der findes et hav mange gratis hosting sites, fx. www.github.com. På dette site kan man gratis kan oprette repositoryies og derefter let kan dele det. 
En anden grund til GIT's store popularitet er måden hvorpå GIT håndtere versions styring. I stedet for at skulle pushe alt til serveren og derefter merge det hele på serveren, har hvert gruppemedlem deres eget repository på deres computer og gruppen har et fælles på deres server. Dette muliggøre offline arbejde og meget bedre merge kontrol. \\

Det blev besluttet at bruge GIT, da GIT er gratis og stor kontrol og frihed.