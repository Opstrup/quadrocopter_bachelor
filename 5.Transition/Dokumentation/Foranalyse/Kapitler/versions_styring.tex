\subsection*{Versions styring}
Til version styring havde vi to værktøjer at vælge imellem. SVN og GIT som begge er to meget gode versions styrings værktøjer, hvilket er med til at give en god struktur på dokumenter og samarbejdet. Af overvejelser omkrig hvilke værktøj vi skulle bruge, prioriterede vi meget højt kontrol og frihed. Med frihed og kontrol  menes der at vi selv kunne styre vores server og let kunne oprettet et ny repository eller redigere i gamle.

\subparagraph*{SVN}
SVN er den ældste teknologi af disse to og det versions værktøj skolen bruger. Vi havde også kendskab til værktøjet fra forrige projekter, men desværre så havde vi også erfaret hvordan denne teknologi tit og ofte giver merge conflicts pga. alt bliver merge på serveren. 

\subparagraph*{GIT}
GIT er det nyeste versions værktøj og er derfor ikke brugt af så mange store virksomheder, men er klart det mest populære værktøj inde for startup virksomheder, mindre virksomheder og open source projekter. En af grundene til dette er at der findes mange gratis hosting sites så som www.github.com hvor man gratis kan oprette repositoryies og derefter let kan dele det med eventuelle kollegaer. En anden grund til GIT's store popularitet er hvordan GIT håndtere versions styring. I stedet for at skulle pushe alt til serveren og derefter merge det hele på serveren, så har hvert gruppe medlem deres eget repository på deres computer og gruppen har et fælles på deres server. Dette muliggøre offline arbejde og meget bedre merge kontrol. \\

Vores valgt faldt på GIT da vi kunne oprette gratis repositoryies på www.github.com, hvilket giver os meget kontrol og frihed. Selvom det var en teknologi som var lige ukendt for så, vurderede vi at arbejdsprocessen var ikke så forskellige som fra SVN. 