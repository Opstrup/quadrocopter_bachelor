\section{Versionsstyring}
I overvejelser omkring værktøjer til versionsstyring blev kontrol og frihed prioriteret højt. Specielt blev mulighed for selv at styre server og nem oprettelse af nyt repository eller redigering af gamle repository's prioriteret højt.
Det blev besluttet at bruge enten SVN eller GIT. Begge er gode versions styrings værktøjer, der bidrager til god struktur og giver overblink over de dokumenter der udarbejdes og deles. 


\textbf{SVN\footnote{http://tortoisesvn.net/}} \\
SVN er et ældre versionsstyrings værktøj og er generelt det værktøj der bruges når projekter udarbejdes på Ingeniørhøjskolen. Det betyder at alle gruppens medlemmer har arbejdet med SVN og derved har kendskab til værktøjets stærke og svage sider. 
Desværre opstår der ofte problemer med SVN, da værktøjet i nogle tilfælde er dårligt til at merge filer, hvilket leder til mange merge conflicts. Disse conflicts er tidskrævende og danner grundlag for megen frustration. 

\textbf{GIT\footnote{https://subversion.apache.org/}} \\
GIT er et nyere versionsstyrings værktøj og er derfor ikke så vel dokumenteret.
Men GIT er et meget populært værktøj, specielt inden for startup virksomheder, mindre virksomheder og open source projekter. 
En af grundene til GIT er et populært værktøj er, at der findes et hav af gratis hosting sites, fx. www.github.com\footnote{https://github.com/}. På dette site kan der oprettes repositories som let kan deles med andre.

En anden grund til GIT's store popularitet er måden hvorpå GIT håndterer versions styring. Ved GIT har hvert gruppemedlem sit eget repository på sin egen computer og en fælles repository på internettet, som er tilgængeligt for alle. Dette muliggøre offline arbejde og meget bedre merge kontrol. \\

Det blev besluttet at bruge GIT, da GIT er gratis og giver stor kontrol og frihed.