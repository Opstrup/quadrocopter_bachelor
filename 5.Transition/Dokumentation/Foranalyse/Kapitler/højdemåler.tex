\section{Højdemåler}

For at sikre mod styrt og sikre quadrocopteren ikke flyver for højt er det nødvendigt, at der på ethvert tidspunkt under flyvning vides hvor højt quadrocopteren befinder sig over jorden. 

Højdemåler vælges ud fra følgende kriterier:  
\begin{enumerate}[label*=\arabic*.]
	\item Præcision.
	\item Rækkevide. 
	\item Vægt. 
\end{enumerate}

\vspace{1cm}

Ud fra kriterierne blev det besluttet at ultralyds afstandsmåleren HC-SR04\footnote{https://www.elextra.dk/main.aspx?page=article&artno=H16466}  var ideel. HC-SR04 er en meget lille og kompakt ultralyds afstandsmåler. Den har rækkevide fra 0.02 – 4.5 meter og er yderst præcis. Ultralydssensoren opererer ved 5 volt og fungerer ved en strømstyrke på 2 mA. HC-SR04 har desuden et interface der gør det nemt at modtage data fra modulet og højdemåleren kan lånes af Ingeniørhøjskolen.

