\section{Kamera}

Under flyvning ønskes det, at der kan tages billeder. Billeder skal sendes til database som systemets bruger kan tilgå. Gruppen prioriterede højt kompakt, lavt strømforbrug og middle billede opløsning for ikke at bruge meget båndbredte. 

\subparagraph*{µCAM}
Et kan der lånes af Ingeniørhøjskolen. Kameraet der hedder µCAM\footnote{http://www.4dsystems.com.au/downloads/micro-CAM/Docs/uCAM-DS-rev7.pdf}. Kameraet kan bruges i forskellige modes. Det påregnes, at kameraet skal bruges i det mode hvor der tages billeder i VGA opløsningen. Opløsningen er 640x480 eller 320x240, hvilket passer meget fint med den ønskede opløsning.

\subparagraph*{JPEG Camera adafruit}
Dette kamera kan købes på www.adafruit.com\footnote{http://www.adafruit.com/products/1386}. Kameraet er super kompakt og med en vægt på kun 3g. Opløsningen er  640x480 eller 320x240. Til kameraet er der udviklet arduino library til styring af kameraet.

\subparagraph*{FlyCamOne eco V2}
Dette kamera kan både optage video og tage billeder. Kameraet kan købes på www.sparkfun.com\footnote{https://www.sparkfun.com/products/11171}. Billede opløsningen er 720x480. Kameraet gemmer film og billeder på et tilhørende SD kort. 





