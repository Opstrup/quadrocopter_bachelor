\section{Kamera}

Under flyvning ønskes det, at der kan tages billeder. Billeder skal sendes til database som systemets bruger kan tilgå. Følgende prioriteres høj: Kamera skal være kompakt, have lavt strømforbrug og tage billeder med lav til middel opløsning for ikke at bruge så meget båndbredde. 

Til projektet er der overvejet nedenstående tre kamera'er:

\textbf{µCAM} \\
Kameraet der hedder µCAM\footnote{http://www.4dsystems.com.au/downloads/micro-CAM/Docs/uCAM-DS-rev7.pdf} og kan lånes af Ingeniørhøjskolen. Kameraet kan bruges i forskellige modes. Det påregnes, at kameraet skal bruges i det mode hvor der tages billeder i VGA opløsningen. Opløsningen er 640x480 eller 320x240, hvilket passer meget fint med den ønskede opløsning.

\textbf{JPEG Camera adafruit} \\
Dette kamera kan købes på www.adafruit.com\footnote{http://www.adafruit.com/products/1386}. Kameraet er super kompakt og med en vægt på kun 3g. Opløsningen er  640x480 eller 320x240. Til kameraet er der udviklet arduino library til styring af kameraet.

\textbf{FlyCamOne eco V2} \\
Dette kamera kan både optage video og tage billeder. Kameraet kan købes på www.sparkfun.com\footnote{https://www.sparkfun.com/products/11171}. Billede opløsningen er 720x480. Kameraet gemmer film og billeder på et tilhørende SD kort. 





