\section{Kamera}

Under flyvning ønskes det, at der kan tages billeder. Billeder skal sendes til database som systemets bruger kan tilgå. Følgende blev prioriteret højt, da der skulle vælges kamera: 

\begin{itemize}
	\item Kamera skal være kompakt
	\item Have lavt strømforbrug
	\item Tage billeder med middel opløsning
\end{itemize}

Til projektet er der overvejet nedenstående fire kamera'er:

\textbf{µCAM} \\
Kameraet der hedder µCAM\footnote{http://www.4dsystems.com.au/downloads/micro-CAM/Docs/uCAM-DS-rev7.pdf} og kan lånes af Ingeniørhøjskolen. Kameraet kan bruges i forskellige modes. Det påregnes, at kameraet skal bruges i det mode hvor der tages billeder i VGA opløsningen. Opløsningen er 640x480 eller 320x240, hvilket passer meget fint med den ønskede opløsning.

\textbf{JPEG Camera adafruit} \\
Dette kamera kan købes på www.adafruit.com\footnote{http://www.adafruit.com/products/1386}. Kameraet er super kompakt og med en vægt på kun 3g. Opløsningen er  640x480 eller 320x240. Til kameraet er der udviklet arduino library til styring af kameraet.

\textbf{FlyCamOne eco V2} \\
Dette kamera kan både optage video og tage billeder. Kameraet kan købes på www.sparkfun.com\footnote{https://www.sparkfun.com/products/11171}. Billede opløsningen er 720x480. Kameraet gemmer film og billeder på et tilhørende SD kort. 

\textbf{Cooking Hacks kamera}
Cooking Hacks' kamera er et lille men simpelt kamera. Kameraet kan monteres på Cooking Hacks' shield, hvilket gør koblingen af systemet lav. Kameraet har en opløsning på 1600 x 1200.

Kameraet der er valgt til projektet er Cooking Hacks' kamera. Cooking Hacks' kamera kan monteres på shieldet, og billederne der tages behøves ikke at komme ned på arduinoen, før de sendes til serveren.



