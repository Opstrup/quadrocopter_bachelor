\documentclass[a4paper,11pt,fleqn,dvipsnames,oneside,openright]{memoir} 	% Openright aabner kapitler paa hoejresider (openany begge)

%%%% PACKAGES %%%%

% ¤¤ Oversaettelse og tegnsaetning ¤¤ %
\usepackage[utf8]{inputenc}					% Input-indkodning af tegnsaet (UTF8)
\usepackage[danish]{babel}					% Dokumentets sprog
\usepackage[T1]{fontenc}					% Output-indkodning af tegnsaet (T1)
\usepackage{ragged2e,anyfontsize}			% Justering af elementer
\usepackage{fixltx2e}						% Retter forskellige fejl i LaTeX-kernen
																			
% ¤¤ Figurer og tabeller (floats) ¤¤ %
\usepackage{graphicx} 						% Haandtering af eksterne billeder (JPG, PNG, EPS, PDF)

\usepackage{subfig}

%\usepackage{eso-pic}						% Tilfoej billedekommandoer paa hver side
%\usepackage{wrapfig}						% Indsaettelse af figurer omsvoebt af tekst. \begin{wrapfigure}{Placering}{Stoerrelse}
\usepackage[space]{grffile}					% Bør gøre det muligt at have mellemrum i filnavne.
\usepackage{multirow}                		% Fletning af raekker og kolonner (\multicolumn og \multirow)
\usepackage{multicol}         	        	% Muliggoer output i spalter
\usepackage{rotating}						% Rotation af tekst med \begin{sideways}...\end{sideways}
\usepackage{colortbl} 						% Farver i tabeller (fx \columncolor og \rowcolor)
\usepackage[usenames,dvipsnames]{xcolor}	% Definer farver med \definecolor. Se mere: http://en.wikibooks.org/wiki/LaTeX/Colors
%\usepackage{flafter}						% Soerger for at floats ikke optraeder i teksten foer deres reference
\let\newfloat\relax 						% Justering mellem float-pakken og memoir
\usepackage{float}							% Muliggoer eksakt placering af floats, f.eks. \begin{figure}[H]
\setlength{\heavyrulewidth}{0.15em}			% Sætter \toprule og \bottomrule til fast størrelse (0.08 er default)
%\setlength{\lightrulewidth}{0.05em}		% Sætter \midrule til fast størrelse (0.05 er default)
\usepackage{array}							% Bruges i forbindelse med \newcolumntype-command under egne commands
\usepackage{pdfpages}						% Bruges så der kan indsættes pdf, som sider (forside for eksempel)
\usepackage{wrapfig}
\usepackage[section]{placeins}				% Indsætter en border så billeder bliver placeret inden for den section de er indsat
\usepackage{lastpage}						% Anvendes til at se total antal sider

% ¤¤ Matematik mm. ¤¤
\usepackage{amsmath,amssymb,stmaryrd} 		% Avancerede matematik-udvidelser
\usepackage{mathtools}						% Andre matematik- og tegnudvidelser
\usepackage{textcomp}                 		% Symbol-udvidelser (f.eks. promille-tegn med \textperthousand )
\usepackage{rsphrase}						% Kemi-pakke til RS-saetninger, f.eks. \rsphrase{R1}
\usepackage[version=3]{mhchem} 				% Kemi-pakke til flot og let notation af formler, f.eks. \ce{Fe2O3}
\usepackage{siunitx}						% Flot og konsistent praesentation af tal og enheder med \si{enhed} og \SI{tal}{enhed}
\sisetup{locale=DE}							% Opsaetning af \SI (DE for komma som decimalseparator) 

% ¤¤ Referencer og kilder ¤¤ %
\usepackage[danish]{varioref}				% Muliggoer bl.a. krydshenvisninger med sidetal (\vref)
\usepackage{natbib}							% Udvidelse med naturvidenskabelige citationsmodeller
\usepackage{xr-hyper}							% Referencer til eksternt dokument med \externaldocument{<NAVN>}
\externaldocument[DokRap-]{../Dokumentationsrapport/Dokumentationsrapport}	% Muliggør eksterne referencer til dokumentationsrapporten
%\usepackage{glossaries}					% Terminologi- eller symbolliste (se mere i Daleifs Latex-bog)



% ¤¤ Misc. ¤¤ %
\usepackage{lipsum}							% Dummy text \lipsum[..]
\usepackage[shortlabels]{enumitem}			% Muliggoer enkelt konfiguration af lister
\usepackage{pdfpages}						% Goer det muligt at inkludere pdf-dokumenter med kommandoen \includepdf[pages={x-y}]{fil.pdf}	
\pdfoptionpdfminorversion=6					% Muliggoer inkludering af pdf dokumenter, af version 1.6 og hoejere
\pretolerance=2500 							% Justering af afstand mellem ord (hoejt tal, mindre orddeling og mere luft mellem ord)

% Kommentarer og rettelser med \fxnote. Med 'final' i stedet for 'draft' udloeser hver note en error i den faerdige rapport.
\usepackage[footnote,draft,danish,silent,nomargin]{fixme}	

%lists
\usepackage{listings}	


%%%% CUSTOM SETTINGS %%%%

% ¤¤ Marginer ¤¤ %
\setlrmarginsandblock{3.5cm}{2.5cm}{*}		% \setlrmarginsandblock{Indbinding}{Kant}{Ratio}
\setulmarginsandblock{2.5cm}{3.0cm}{*}		% \setulmarginsandblock{Top}{Bund}{Ratio}
\checkandfixthelayout 						% Oversaetter vaerdier til brug for andre pakker

%	¤¤ Afsnitsformatering ¤¤ %
\setlength{\parindent}{0mm}           		% Stoerrelse af indryk
\setlength{\parskip}{3mm}          			% Afstand mellem afsnit ved brug af double Enter
\linespread{1,1}							% Linie afstand
\newcommand{\tab}{\hspace*{2em}}			% ved \tab{} indrykkes det i klammerne ind
\usepackage{titlesec}							%Muliiggøre ændring af sections i alle lag
\titleformat*{\section}{\LARGE\bfseries\color{NavyBlue}}		%section = størst
\titleformat*{\subsection}{\Large\bfseries\color{RoyalBlue}}		%sub og subsub har samme størrelse
\titleformat*{\subsubsection}{\Large\bfseries}
\titleformat*{\paragraph}{\large\bfseries}		%Benyttes umiddelbart ikke
\titleformat*{\subparagraph}{\large\bfseries}	%Benyttes umiddelbart ikke

% ¤¤ Litteraturlisten ¤¤ %
\bibpunct[,]{[}{]}{;}{a}{,}{,} 				% Definerer de 6 parametre ved Harvard henvisning (bl.a. parantestype og seperatortegn)
\bibliographystyle{bibtex/harvard}			% Udseende af litteraturlisten.

% ¤¤ Indholdsfortegnelse ¤¤ %
\setsecnumdepth{subsubsection}		 		% Dybden af nummerede overkrifter (part/chapter/section/subsection)
\maxsecnumdepth{subsection}					% Dokumentklassens graense for nummereringsdybde
\settocdepth{subsubsection} 				% Dybden af indholdsfortegnelsen

% ¤¤ Lister ¤¤ %
\setlist{
  topsep=-5pt,								% Vertikal afstand mellem tekst og listen	Default: 0
  itemsep=-1ex,								% Vertikal afstand mellem items
} 

% ¤¤ Visuelle referencer ¤¤ %
\usepackage[colorlinks]{hyperref}			% Danner klikbare referencer (hyperlinks) i dokumentet.
\hypersetup{colorlinks = true,				% Opsaetning af farvede hyperlinks (interne links, citeringer og URL)
    linkcolor = black,
    citecolor = black,
    urlcolor = black
}

% ¤¤ Opsaetning af figur- og tabeltekst ¤¤ %
\usepackage{caption}
\captionnamefont{\small\bfseries\itshape}	% Opsaetning af tekstdelen ('Figur' eller 'Tabel')
\captiontitlefont{\small}					% Opsaetning af nummerering
\captiondelim{. }							% Seperator mellem nummerering og figurtekst
\hangcaption								% Venstrejusterer flere-liniers figurtekst under hinanden
\captionsetup{width=\linewidth,labelfont={bf,it}}
\setlength{\abovecaptionskip}{5pt}			% Afstand over figurteksten
\setlength{\belowcaptionskip}{-12pt}		% Afstand under figurteksten
		
% ¤¤ Navngivning ¤¤ %
\addto\captionsdanish{
	\renewcommand\appendixname{Appendiks}
	\renewcommand\contentsname{Indholdsfortegnelse}	
	\renewcommand\appendixpagename{Appendiks}
	\renewcommand\appendixtocname{Appendiks}
	\renewcommand\cftchaptername{\chaptername~}				% Skriver "Kapitel" foran kapitlerne i indholdsfortegnelsen
	\renewcommand\cftappendixname{\appendixname~}			% Skriver "Appendiks" foran appendiks i indholdsfortegnelsen
}

% ¤¤ Kapiteludssende ¤¤ %
\definecolor{chapnumcolor}{RGB}{23,54,93}		% Definerer en farve til brug til kapiteludseende
\definecolor{chapfontcolor}{RGB}{29,69,118}
\newif\ifchapternonum

\makechapterstyle{jenor}{					% Definerer kapiteludseende frem til ...
  \renewcommand\beforechapskip{0pt}
  \renewcommand\printchaptername{}
  \renewcommand\printchapternum{}
  \renewcommand\printchapternonum{\chapternonumtrue}
  \renewcommand\chaptitlefont{\fontfamily{pbk}\fontseries{db}\fontshape{n}\fontsize{25}{35}\selectfont\raggedleft\color{chapfontcolor}}
  \renewcommand\chapnumfont{\fontfamily{pbk}\fontseries{m}\fontshape{n}\fontsize{1in}{0in}\selectfont\color{chapnumcolor}}
  \renewcommand\printchaptertitle[1]{%
    \noindent
    \ifchapternonum
    \begin{tabularx}{\textwidth}{X}
    {\let\\\newline\chaptitlefont ##1\par} 
    \end{tabularx}
    \par\vskip-2.5mm\hrule
    \else
    \begin{tabularx}{\textwidth}{Xl}
    {\parbox[b]{\linewidth}{\chaptitlefont ##1}} & \raisebox{-15pt}{\chapnumfont \thechapter}
    \end{tabularx}
    \par\vskip2mm\hrule
    \fi
  }
}											% ... her

\chapterstyle{jenor}						% Valg af kapiteludseende - Google 'memoir chapter styles' for alternativer

% ¤¤ Sidehoved ¤¤ %

\makepagestyle{AAU}							% Definerer sidehoved og sidefod udseende frem til ...
\makepsmarks{AAU}{%
	\createmark{chapter}{left}{shownumber}{}{. \ }
	\createmark{section}{right}{shownumber}{}{. \ }
	\createplainmark{toc}{both}{\contentsname}
	\createplainmark{lof}{both}{\listfigurename}
	\createplainmark{lot}{both}{\listtablename}
	\createplainmark{bib}{both}{\bibname}
	\createplainmark{index}{both}{\indexname}
	\createplainmark{glossary}{both}{\glossaryname}
}
\nouppercaseheads											% Ingen Caps oenskes

\makeevenhead{AAU}{Test}{}{\leftmark}					% Definerer lige siders sidehoved (\makeevenhead{Navn}{Venstre}{Center}{Hoejre})
\makeoddhead{AAU}{\rightmark}{}{Ingeniørhøjskolen, Aarhus Universitet}		% Definerer ulige siders sidehoved (\makeoddhead{Navn}{Venstre}{Center}{Hoejre})
\makeevenfoot{AAU}{\thepage}{}{}							% Definerer lige siders sidefod (\makeevenfoot{Navn}{Venstre}{Center}{Hoejre})
\makeoddfoot{AAU}{}{}{\thepage}								% Definerer ulige siders sidefod (\makeoddfoot{Navn}{Venstre}{Center}{Hoejre})
\makeheadrule{AAU}{\textwidth}{0.5pt}						% Tilfoejer en streg under sidehovedets indhold
\makefootrule{AAU}{\textwidth}{0.5pt}{1mm}					% Tilfoejer en streg under sidefodens indhold

\copypagestyle{AAUchap}{AAU}								% Sidehoved for kapitelsider defineres som standardsider, men med blank sidehoved
\makeoddhead{AAUchap}{}{}{}
\makeevenhead{AAUchap}{}{}{}
\makeheadrule{AAUchap}{\textwidth}{0pt}
\aliaspagestyle{chapter}{AAUchap}							% Den ny style vaelges til at gaelde for chapters
															% ... her
															
\pagestyle{AAU}												% Valg af sidehoved og sidefod



% Opsætning af source code import
% \lstinputlisting{sti../navn.endelse}
\lstset{
  language=C,                		% choose the language of the code
  numbers=left,                   	% where to put the line-numbers
  stepnumber=1,                   	% the step between two line-numbers.        
  numbersep=5pt,                  	% how far the line-numbers are from the code
  backgroundcolor=\color{white},  	% choose the background color. You must add \usepackage{color}
  showspaces=false,               	% show spaces adding particular underscores
  showstringspaces=false,         	% underline spaces within strings
  showtabs=false,                 	% show tabs within strings adding particular underscores
  tabsize=2,                      	% sets default tabsize to 2 spaces
  captionpos=b,                   	% sets the caption-position to bottom
  breaklines=true,                	% sets automatic line breaking
  breakatwhitespace=true,         	% sets if automatic breaks should only happen at whitespace
  title=\lstname,                 	% show the filename of files included with \lstinputlisting;
  emph={ uint8, uint16, void }, emphstyle={\color{blue}}	% tilføj variabel hvis de skal markeres blå
}






%%%% CUSTOM COMMANDS %%%%

% ¤¤ Billede hack ¤¤ %
\newcommand{\figur}[4]{
		\begin{figure}[H] \centering
			\includegraphics[width=#1\textwidth]{Billeder/#2}
			\caption{#3}\label{#4}
		\end{figure} 
}


% ¤¤ Venstre orienterer al tekst i p{Ycm} ¤¤ %
\newcolumntype{x}[1]{%
>{\raggedright\hspace{0pt}}p{#1}}

% ¤¤ Newline til x{} ¤¤ %
% \\ virker åbenbart ikke når man selv laver en columntype... :(
\newcommand{\tn}{\tabularnewline}

% ¤¤ Newline til x{} ¤¤ %
% \\ virker åbenbart ikke når man selv laver en columntype... :(
\newcommand{\tnhl}{\tabularnewline\hline}



% ¤¤ Nyt environment til indsættelse af A3-størrelse figurer
\newenvironment{A3Figure}
{
	\cleardoublepage
	\pageaiii
	\setlength{\pdfpagewidth}{\paperheight} % Change the pdf page to A3 height
	\setlength{\pdfpageheight}{\paperwidth} % Change the pdf height to A3 width
	\setlength{\textwidth}{\paperheight - \the\spinemargin-\the\foremargin} % Change the textwidth
}
{
	\cleardoublepage	
}

% funktions beskrivelse void argument%
\newcommand{\funcDescrip}[2]{
\textbf{{\color{blue} #1} #2({\color{blue} void})} \\
}

% funktions beskrivelse 1 argument%
\newcommand{\funcDescripOne}[4]{
\textbf{{\color{blue} #1} #2({\color{blue} #3} #4)} \\
}

% funktions beskrivelse 2 argument%
\newcommand{\funcDescripTwo}[6]{
\textbf{{\color{blue} #1} #2({\color{blue} #3} #4, {\color{blue} #5} #6)} \\
}

% opsætning af funktions tabel %
\newcommand{\funcTabel}[4]{
\begin{tabular}{p{0.2cm}p{3cm}p{11cm}} \hline
	&	\textbf{Description:} 		& 	#1	\\
	&	\textbf{Parameters:} 		& 	#2	\\
	&	\textbf{Return Value:}		& 	#3	\\
	&	\textbf{Side Effects:}		&	#4	\\
\end{tabular}
}


% ¤¤ Units i math-environments ¤¤ %
\newcommand{\mathUnit}[2]{\mathrm{\si{#1}{#2}}}







% ¤¤ Pæn opsætning af titelblad-dele ¤¤ %
% ¤¤ Husk at ændre dato i senere projekter ¤¤ %
\newcommand{\titelblad}[2]{
\begin{tabular}[ht]{x{7cm}x{7cm}}
\textbf{Navn: } #1		&\textbf{Studienummer: } #2	\tn
\textbf{Dato} 2013-05-31	\tn
\multicolumn{2}{l}{\textbf{Underskrift: }\line(1,0){340}}
\end{tabular}
}


% ¤¤ Specielle tegn ¤¤ %
\newcommand{\grader}{^{\circ}\text{C}}   % Grader C, virker kun i math-environments
\newcommand{\gr}{^{\circ}}
\newcommand{\g}{\cdot}		% Gange i math-environments
\newcommand{\grC}{$^{\circ}\mathrm{C}$}		% Grader C, uden for math-environments

%%%% ORDDELING %%%%

\hyphenation{}

%%%Indsat af Søren%%%
\usepackage{listings}
\usepackage{color}
 
\definecolor{dkgreen}{rgb}{0,0.6,0}
\definecolor{gray}{rgb}{0.5,0.5,0.5}
\definecolor{mauve}{rgb}{0.58,0,0.82}
 
\lstset{ %
  language=Octave,                % the language of the code
  basicstyle=\footnotesize,           % the size of the fonts that are used for the code
  numbers=left,                   % where to put the line-numbers
  numberstyle=\tiny\color{gray},  % the style that is used for the line-numbers
  stepnumber=2,                   % the step between two line-numbers. If it's 1, each line 
                                  % will be numbered
  numbersep=5pt,                  % how far the line-numbers are from the code
  backgroundcolor=\color{white},      % choose the background color. You must add \usepackage{color}
  showspaces=false,               % show spaces adding particular underscores
  showstringspaces=false,         % underline spaces within strings
  showtabs=false,                 % show tabs within strings adding particular underscores
  frame=single,                   % adds a frame around the code
  rulecolor=\color{black},        % if not set, the frame-color may be changed on line-breaks within not-black text (e.g. comments (green here))
  tabsize=2,                      % sets default tabsize to 2 spaces
  captionpos=b,                   % sets the caption-position to bottom
  breaklines=true,                % sets automatic line breaking
  breakatwhitespace=false,        % sets if automatic breaks should only happen at whitespace
  title=\lstname,                   % show the filename of files included with \lstinputlisting;
                                  % also try caption instead of title
  keywordstyle=\color{blue},          % keyword style
  commentstyle=\color{dkgreen},       % comment style
  stringstyle=\color{mauve},         % string literal style
  escapeinside={\%*}{*)},            % if you want to add LaTeX within your code
  morekeywords={*,...},              % if you want to add more keywords to the set
  deletekeywords={...}              % if you want to delete keywords from the given language
}