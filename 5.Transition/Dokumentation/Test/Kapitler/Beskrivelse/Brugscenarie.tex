\section{Brugscenarie}

Et almindeligt brugsscenarie for den autonome overvågnings drone vil være at få tildelt nogle koordinater, typisk omkring en virksomhed eller et område, som skal overvåges.
Brugeren indtaster de lokationer dronen skal overvåge, hvorefter disse uploades. 
Brugeren har mulighed for at vælge mellem nogle forskellige indstillinger, så som flyvehøjde, højden billeder skal tages i og antal steder der skal overvåges.
Under flyvningen kan bruger godkende billederne, hvis billederne ikke godkendes, ændrer dronen positionen, hvorefter et nyt billede tages og sendes til godkendelse. Hvis ikke brugeren får godkendt billedet indenfor tidsgrænsen, vil det blive betragtet som at billedet godkendes og dronen flyver videre til næste koordinat.

\section{Prioritering}

\begin{table}[H]
	\centering
		\begin{tabular}{|l|l|p{7 cm}|} 
		\hline
			Område & Prioritering & Kommentar \\ \hline
			Sikkerhed 		& 5 	& Sikkerheden prioriteres højst, idet dronens propeller roterer med en hastighed der kan skade personer og dyr.   \\ \hline
			
			Pålidelighed 	& 4 	& Hele systemet skal være pålidelig, da den under flyvning ikke må udsætte mennesker og dyr for fare.  \\ \hline
			
			Pris 			& 3 	& Prisen er mindre vigtig, da dette er et udviklingsprojekt med henblik på videreudvikling.   \\ \hline
			
			Brugervenlighed & 3 	& Systemet skal ikke kunne betjenes af alle, så brugervenligheden er ikke den vigtigste prioritering. \\ \hline
		\end{tabular}
	\caption{Prioriteringsliste}
\end{table}
