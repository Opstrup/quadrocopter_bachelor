\chapter{Funktionelle krav}

\vspace{-1cm}

\section*{Formål}
\vspace{-0.5cm}

Dette afsnit specificerer accepttests af systemets funktionelle krav. 

Hvis der under accepttesten opstår fejl, der umuliggør fortsat test af efterfølgende testcases afbrydes denne test. Hvis der opstår fejl i enkelte testcases; men fortsat accepttest er mulig, underkendes den enkelte test og accepttesten fortsættes.

Såfremt en test afbrydes eller en testcase underkendes eller ikke kan gennemføres, skrives den eller de step der ikke kan gennemføres angives de i kapitlet \textbf{Fejl og mangler}. 


\section*{Testspecifikation}
\vspace{-0.5cm}
Software der skal testes:
\begin{table}[H]
	\centering
		\begin{tabular}{|c|c|c|c|}
			\hline
			Software & Version & Release dato & Bemærkning \\ \hline
			Server & Rev. 226b85b & 23/11-2014 & \\ \hline			
			Webapplikation & Rev. 7969aac & 04/12-2014 &\\ \hline
		\end{tabular}
	\caption{Software til test}
\end{table}

Hardware der skal testes:
\begin{table}[H]
	\centering
		\begin{tabular}{|c|c|c|c|}
			\hline
			Hardware & Version & Release dato & Bemærkning \\ \hline
			Kompas 			& Rev. 9d3ec7b & 17/11-2014 &  \\ \hline			
			Højdemåler 		& Rev. 9d3ec7b & 17/11/2014 &  \\ \hline
			3G/GPS modul 	& Rev. 51ec5ac & 7/11-2014 &  \\ \hline
			Switch board 	& Rev. 9d3ec7b & 17/11/2014 &  \\ \hline
		\end{tabular}
	\caption{Hardware til test}
\end{table}

\vspace{-0.5cm}
\section*{Testprocedure}
\vspace{-0.5cm}
De individuelle use cases og scenarier i kravspecifikationen testes i enkelte test cases. 

\begin{itemize}
	\item Hvis et teststep gennemføres fejlfrit markeres dette med ”Godkendt” i feltet ”resultat” for det pågældende test step.

	\item Hvis et teststep gennemføres med ubetydelige fejl, markeres dette med ”(OK)” i feltet ”resultat” for det pågældende test step.
	
	\item Hvis et teststep ikke kan gennemføres eller gennemføres med betydelige fejl, markeres dette med "Ikke godkendt".
	
\end{itemize}

\newpage

\section{Test case 1: Start drone}
Use case under test: UC 1: Start drone.\\
Forudsætninger:	Ingen.

\textbf{Hovedscenarie}
\begin{table}[H]
	\centering
		\begin{tabular}{|l|p{5 cm}|p{5 cm}|p{2.5 cm}|} 
		\hline
			\textbf{Step} & \textbf{Handling} & \textbf{Forventet resultat} & \textbf{Resultat} \\ \hline
			1 & Bruger tilslutter batteri og dronen tændes. & Systemet tilføres forsyning og ESC'er signalerer de er \newline forbundet til forsyning. & Godkendt.   \\ \hline
			2 & Main controller initialiseres. & Main controller startes. & Godkendt.   \\ \hline
			3 & GPS initialiseres og dronens nuværende GPS position \newline opdateres. & Nuværende GPS position \newline opdateres og gemmes lokalt på main controller. & Godkendt.   \\ \hline
			4 & 3G initialiseres og dronen opretter forbindelse til \newline 3G-netværket. & Dronen tilsluttes det mobile \newline 3G-netværk. &  Godkendt. \\ \hline
			5 & Dronens nuværende GPS \newline position sendes til server. & På webapplikation vises \newline drones nuværende position og at dronen er online. & Godkendt. \\ \hline
		\end{tabular}
\end{table}

\textbf{Extension 1: Forbindelse kan ikke oprettes.}
\begin{table}[H]
	\centering
		\begin{tabular}{|l|p{5 cm}|p{5 cm}|p{2.5 cm}|} 
		\hline
			\textbf{Step} & \textbf{Handling} & \textbf{Forventet resultat} & \textbf{Resultat} \\ \hline
			a & Systemet indikerer fejl og bruger genstarter systemet. & Systemet genstartes og \newline forbindelse oprettes. & Godkendt. \\ \hline
		\end{tabular}
\end{table}

\newpage
\section{Test case 2: Ny flyveopsætning}
Use case under test: UC 2: Ny flyveopsætning.\\
Forudsætninger:	Bruger er oprettet i systemet og UC\#1 er succesfuld gennemført.

\textbf{Hovedscenarie}
\begin{table}[H]
	\centering
		\begin{tabular}{|l|p{5 cm}|p{5 cm}|p{2.5 cm}|}
		\hline
			\textbf{Step} & \textbf{Handling} & \textbf{Forventet resultat} & \textbf{Resultat} \\ \hline
			1 & Bruger logger på \newline webapplikation. & Login lykkes, og webapplikationens forside vises. & Godkendt. \\ \hline
			2 & Fra forsiden navigerer bruger til flyveopsætning. & Flyveopsætnings siden vises. & Godkendt. \\ \hline
			3 & Bruger laver en ny \newline flyveopsætning og vælger: \newline
				- GPS lokationer der \newline skal flyves til. \newline
				- Om der skal tages billeder ved de valgte lokationer. \newline
				- Højde billeder skal tages fra. \newline
				- Generel flyvehøjde.				
				 & Flyveopsætning klargjort & Godkendt. \\ \hline
			4 & Bruger gemmer flyveopsætning og gør den tilgængelig på server. & Flyveopsætning gøres \newline tilgængelig på server & Ikke godkendt. \\ \hline
		\end{tabular}
\end{table}

\textbf{Extension 1: Fejl i login.}
\begin{table}[H]
	\centering
		\begin{tabular}{|l|p{5 cm}|p{5 cm}|p{2.5 cm}|} 
		\hline
			\textbf{Step} & \textbf{Handling} & \textbf{Forventet resultat} & \textbf{Resultat} \\ \hline
			a & Bruger føres tilbage til login. & Login kan påny forsøges. & Godkendt. \\ \hline
		\end{tabular}
\end{table}

\textbf{Extension 2: Der laves ikke ny flyveopsætning.}
\begin{table}[H]
	\centering
		\begin{tabular}{|l|p{5 cm}|p{5 cm}|p{2.5 cm}|} 
		\hline
			\textbf{Step} & \textbf{Handling} & \textbf{Forventet resultat} & \textbf{Resultat} \\ \hline
			a & Gemt flyveopsætning bruges. & Flyveopsætning klargjort. & Ikke godkendt.\\ \hline
		\end{tabular}
\end{table}

\newpage

\section{Test case 3: Flyv til position}
Use case under test: UC 3: Flyv til position.\\
Forudsætninger:	UC\#1 og UC\#2 er succesfuld gennemført.

\textbf{Hovedscenarie}
\begin{table}[H]
	\centering
		\begin{tabular}{|l|p{5 cm}|p{5 cm}|p{2.5 cm}|} 
		\hline
			\textbf{Step} & \textbf{Handling} & \textbf{Forventet resultat} & \textbf{Resultat} \\ \hline
			1 & Drone henter flyveopsætning fra server. & Der oprettes forbindelse \newline mellem server og drone \newline og flyveopsætning sendes. & Godkendt. \\ \hline
			2 & Nuværende position \newline opdateres. & Ingen handling  &  Godkendt. \\ \hline
			3 & Flyvehøjde tilpasses. & Flyvehøjde justeres & Godkendt. \\ \hline
			4 & Flyveorientering tilpasses. & Orientering justeres & Ikke godkendt. \\ \hline
			5 & Drone flyver mod \newline ønsket position. & Drone nærmer sig \newline ønsket position  &  Ikke godkendt. \\ \hline
			6 & Ønsket position er nået. & Ønsket position er nået. & Ikke godkendt. \\ \hline
		\end{tabular}
\end{table}

\textbf{Extension 1: Dronen henter ikke flyveopsætning.}
\begin{table}[H]
	\centering
		\begin{tabular}{|l|p{5 cm}|p{5 cm}|p{2.5 cm}|}
		\hline
			\textbf{Step} & \textbf{Handling} & \textbf{Forventet resultat} & \textbf{Resultat} \\ \hline
			a & Flyvning med en anden flyveopsætning er aktiv. & UC \#3 fortsættes med den aktive flyveopsætning. & Godkendt. \\ \hline
		\end{tabular}
\end{table}

\textbf{Extension 2: Ugyldig GPS koordinat.}
\begin{table}[H]
	\centering
		\begin{tabular}{|l|p{5 cm}|p{5 cm}|p{2.5 cm}|} 
		\hline
			\textbf{Step} & \textbf{Handling} & \textbf{Forventet resultat} & \textbf{Resultat} \\ \hline
			a & Drone går i fejlmode \#1. & Forsøger at finde gyldig GPS koordinat, mislykkes dette lander dronen. & Ikke godkendt.\\ \hline
		\end{tabular}
\end{table}

\textbf{Extension 3: Ugyldig flyvehøjde.}
\begin{table}[H]
	\centering
		\begin{tabular}{|l|p{5 cm}|p{5 cm}|p{2.5 cm}|}
		\hline
			\textbf{Step} & \textbf{Handling} & \textbf{Forventet resultat} & \textbf{Resultat} \\ \hline
			a & Drone går i fejlmode \#2. & Forsøger at finde gyldig \newline flyvehøjde, mislykkes dette lander dronen. & Ikke godkendt. \\ \hline
		\end{tabular}
\end{table}

\newpage

\section{Test case 4: Billede af position}
Use case under test: UC 4: Billede af position.\\
Forudsætninger:	UC\#3 er succesfuld gennemført.

\textbf{Hovedscenarie}
\begin{table}[H]
	\centering
		\begin{tabular}{|l|p{5 cm}|p{5 cm}|p{2.5 cm}|} 
		\hline
			\textbf{Step} & \textbf{Handling} & \textbf{Forventet resultat} & \textbf{Resultat} \\ \hline
			1 & Drone tager et billede af \newline nuværende position. & Kamera aktiveres og der \newline tages et billede. & Ikke godkendt. \\ \hline
			2 & Billedet sendes til server. & Billedet sendes til server gøres tilgængelig på \newline webapplikationen & Ikke godkendt. \\ \hline
			3 & Bruger giver accept af billedet via webapplikation. & Billedet gemmes i database  & Ikke godkendt. \\ \hline

		\end{tabular}
\end{table}


\textbf{Extension 1: Bruger beder om nyt billede.}
\begin{table}[H]
	\centering
		\begin{tabular}{|l|p{5 cm}|p{5 cm}|p{2.5 cm}|} 
		\hline
			\textbf{Step} & \textbf{Handling} & \textbf{Forventet resultat} & \textbf{Resultat} \\ \hline
			a & Drone instrueres til at \newline ændre flyvehøjde, orientering eller position. &  Flyvehøjde, orientering eller position ændres. & Ikke godkendt. \\ \hline
		\end{tabular}
\end{table}

\textbf{Extension 2: Bruger svarer ikke inden for tidsgrænsen.}
\begin{table}[H]
	\centering
		\begin{tabular}{|l|p{5 cm}|p{5 cm}|p{2.5 cm}|} 
		\hline
			\textbf{Step} & \textbf{Handling} & \textbf{Forventet resultat} & \textbf{Resultat} \\ \hline
			a & Drone får automatisk tildelt accept. & Drone påbegynder flyvning mod næste GPS lokation. & Ikke godkendt. \\ \hline
		\end{tabular}
\end{table}

\newpage
\section{Test case 5: Vis tidligere flyvning}
Use case under test: UC 5: Vis tidligere flyvning.\\
Forudsætninger:	Bruger er oprettet i systemet.

\textbf{Hovedscenarie}
\begin{table}[H]
	\centering
		\begin{tabular}{|l|p{5 cm}|p{5 cm}|p{2.5 cm}|} 
		\hline
			\textbf{Step} & \textbf{Handling} & \textbf{Forventet resultat} & \textbf{Resultat} \\ \hline
			1 & Bruger logger på \newline webapplikation. & Login lykkes og webapplikations forside vises. & Ikke godkendt.  \\ \hline
			2 & Fra forsiden navigerer \newline bruger til historik. & En oversigt over tidligere \newline flyvninger vises. & Ikke godkendt. \\ \hline
			3 & Bruger vælger en specifik \newline tidligere flyvning. &  Billeder, video og flyverute \newline tilhørende valgte flyvning \newline vises. &  Ikke godkendt. \\ \hline			
		\end{tabular}
\end{table}

\textbf{Extension 1: Fejl i login.}
\begin{table}[H]
	\centering
		\begin{tabular}{|l|p{5 cm}|p{5 cm}|p{2.5 cm}|} 
		\hline
			\textbf{Step} & \textbf{Handling} & \textbf{Forventet resultat} & \textbf{Resultat} \\ \hline
			a & Bruger føres tilbage til login. & Login kan påny forsøges. & Ikke godkendt. \\ \hline
		\end{tabular}
\end{table}

\section{Test case 6: Anti kollision}
Use case under test: UC 6: Anti kollision.\\
Forudsætninger:	UC\#3 er igangværende.

\textbf{Hovedscenarie}
\begin{table}[H]
	\centering
		\begin{tabular}{|l|p{5 cm}|p{5 cm}|p{2.5 cm}|} 
		\hline
			\textbf{Step} & \textbf{Handling} & \textbf{Forventet resultat} & \textbf{Resultat} \\ \hline
			1 & Anti kollisions system \newline detekterer en forhindring. & Bremser dronens fremdrift. & Ikke godkendt. \\ \hline
			2 & Undvigelsesmanøvre udføres. & Drone passere forhindring og flyver videre. & Ikke godkendt.\\ \hline			
		\end{tabular}
\end{table}

\textbf{Extension 1: Forhindringen kan ikke undviges.}
\begin{table}[H]
	\centering
		\begin{tabular}{|l|p{5 cm}|p{5 cm}|p{2.5 cm}|} 
		\hline
			\textbf{Step} & \textbf{Handling} & \textbf{Forventet resultat} & \textbf{Resultat} \\ \hline
			a & Drone går i fejlmode \#3. & Forsøger at finde måde at passere forhindring. Mislykkes det lander dronen. & Ikke godkendt. \\ \hline
		\end{tabular}
\end{table}


