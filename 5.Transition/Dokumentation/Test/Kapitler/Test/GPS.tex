\section{GPS}

Der er to mål med enhedstest af GPS. Først og fremmest testes hvorvidt GPS'en virker, desuden testes hvor præcis GPS'en er. I kravspecifikationen er det defineret hvor stor en afvigelse GPS modulet må have, i testen undersøges det om afvigelse overholdes. 

Hvis der hentes GPS koordinater lige efter GPS'en tændes gøres der brug af \textit{cold start}. Ved brug af cold start fås data fra GPS'en med det samme GPS'en har forbindelse til GPS satellitter, men dataen er upålidelig og upræcis. Derfor gøres der altid brug af \textit{hot start}, både under test og når GPS'en bruges på dronen. Når der bruges hot start er GPS'en i udgangspunkt i forbindelse med GPS-satelitter og er derfor mere præcis. 

Under test tændes modulet og får 30 sekunder til opstart. Efter de 30 sekunder er gået hentes data ud fra GPS modulet. Raw data fra GPS kommer i formatet Degrees/Minutes, og ved omformning fås decimal degrees.

For at beregne GPS'en præcision beregnes forskel på nuværende position og den position der angives af GPS'en. Resultat af enhedstesten er vist i tabellen nedenfor. 


\begin{table}[H]
\begin{tabular}{| p{5.3cm}| p{5.3cm}| p{3cm}|}
\hline
\textbf{GPS moduls koordinater} & \textbf{Aktuelle position} & \textbf{Afvigelse}\\\hline
Latitude:10.191518 \newline Longitude: 56.171863 & Latitude: 10.191560 \newline Longitude: 56.171896 & 4 meter\\\hline

\end{tabular}
\caption{GPS modul}
\label{tab:GPS_modul}
\end{table}

