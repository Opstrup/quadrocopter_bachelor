\section{Afstandssensor}

I denne sektion beskrives hvordan ultralyds sensoren HC-SR04 er testet for at fastslå hvor pålidelig den er. Den udførte test er foregået ved at lave gentagende målinger af afstand mellem ultralyds sensor og en væg. For hver deltest er der foretaget 100 målinger og maksimal værdi, minimal værdi og gennemsnit fundet. 

Ultralyds sensoren bruges både til højdemåling og antikollison. Testen bruges til at kontrollere hvorvidt sensoren kan opfylde krav om nøjagtighed til højdemåling\footnote{Se ikke funktionelle krav i \textit{Kravspecifikation}}. 

Undervejs i testen ændres vinklen mellem ultralyds sensor og væg, hvilket indirekte betyder afstande også ændres til væggen. Det formodes at sensoren fungerer bedst når der er en 90 graders vinkel mellem sensor og væg. 

På figur \ref{fig:ultra_testopstilling} vises en skitse af den anvendte testopstilling og tabellen nederst på siden viser resultater af den udførte test.

\begin{figure}[H]
\centering
\includegraphics[width=1\textwidth]{Billeder/Test/ultrasound.png}
\caption{Skitse af testopstilling}
\label{fig:ultra_testopstilling}
\end{figure}

\vspace{0.5cm}

\begin{table}[H]
\begin{tabular}{| p{2.5cm}| p{2.5cm}| p{2.5cm}| p{2.5cm}| p{2.5cm}|}
\hline
Vinkel & Afstand & Maks måling & Min måling  & Gns værdi \\ \hline
90 & 50 cm & 49.0 cm & 48.0 cm  & 48.02 cm \\ \hline
70 & 56 cm & 49.0 cm & 49.0 cm  & 49.0 cm \\ \hline
50 & 62 cm & 50.0 cm & 51.0 cm  & 50.0 cm \\ \hline

\end{tabular}
\caption{Test ultralydssensor}
\label{tab:Ultralyds_test}
\end{table}


