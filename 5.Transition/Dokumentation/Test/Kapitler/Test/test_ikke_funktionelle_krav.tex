\chapter{Accepttest \\ Ikke funktionelle krav}

\section*{Formål}
\vspace{-0.5cm}
Dette afsnit specificerer accepttests af systemets ikke funktionelle krav. Indledningsvis gennemgås krav til systemet som helhed og senere testes krav der har henblik på specifikke enheder i systemet. 

Såfremt krav ikke kan opfyldes angives de i kapitlet \textbf{Fejl og mangler}.

\section*{Testprocedure}
\vspace{-0.5cm}
De individuelle use cases og scenarier i kravspecifikationen testes i enkelte test cases. 

\begin{itemize}
	\item Hvis et teststep gennemføres fejlfrit markeres dette med ”Godkendt” i feltet ”resultat” for det pågældende test step.

	\item Hvis et teststep gennemføres med ubetydelige fejl, markeres dette med ”(OK)” i feltet ”resultat” for det pågældende test step.
	
	\item Hvis et teststep ikke kan gennemføres eller gennemføres med betydelige fejl, markeres dette med "Ikke godkendt".
	
\end{itemize}

\vspace{1.5cm}


    
    \begin{tabular}{|p{1. cm}|p{3.2 cm}|p{3.2 cm}|p{3.2 cm}|p{2.2 cm}|}
			\hline
			\multicolumn{5}{|l|}{\textbf{{\large Generelle krav}}}\\ \hline
			\textbf{Krav nr.} & \textbf{Krav} & \textbf{Test} & \textbf{Forventet \newline resultat} & 			
			\textbf{Resultat} \\ \hline
			
			1.1 & Kommunikation mellem drone og webapplikation skal foregå trådløst.
				& Opsætning sendes til drone via webapplikation.
				& Den sendte opsætning modtages.
				& Godkendt. \\ \hline

			1.2 & Trådløs kommunikation benytter 3G protokol eller ældre. 
				& Der undersøges hvilken slags mobilnet 3G-modul er forbundet til.
				& Det verificeres at protokollen passer.
				& Godkendt. \\ \hline
			
			1.3 & Højdemåler skal måle højde $\pm$ 10 cm.
				& Højdemåler placeres i en kendt afstand til en væg.
				& Afstanden måles med korrekt længde.
				& Godkendt.\\ \hline						
		\end{tabular}

\vspace{2cm}

    
    \begin{tabular}{|p{1. cm}|p{3.2 cm}|p{3.2 cm}|p{3.2 cm}|p{2.2 cm}|}
			\hline
			\multicolumn{5}{|l|}{\textbf{{\large Krav til server}}}\\ \hline
			\textbf{Krav nr.} & \textbf{Krav} & \textbf{Test} & \textbf{Forventet \newline resultat} & 			
			\textbf{Resultat} \\ \hline
			
			2.1 & Indholder database med billeder og flyveruter.
				& Databasen tilgås, billeder og flyveruter vises.
				& Databasen indeholder billeder og flyveruter.
				& Godkendt. \\ \hline

			2.2 & Indholder database med brugere.
				& Databasen tilgås og en liste med alle brugere vises.
				& De oprettede brugere vises.
				& Godkendt. \\ \hline		
				
			2.3 & Indholder database med flyveopsætninger.
				& Databasen tilgås og en liste med flyveopsætninger vises.
				& De oprettede flyveopsætninger vises.
				& Godkendt. \\ \hline		
		\end{tabular}

\vspace{2.5cm}

    
    \begin{tabular}{|p{1. cm}|p{3.2 cm}|p{3.2 cm}|p{3.2 cm}|p{2.2 cm}|}
			\hline
			\multicolumn{5}{|l|}{\textbf{{\large Krav til webapplikation}}}\\ \hline
			\textbf{Krav nr.} & \textbf{Krav} & \textbf{Test} & \textbf{Forventet \newline resultat} & 			
			\textbf{Resultat} \\ \hline
			
			3.1 & Webapplikation skal kunne tilgås via både computere og telefoner.
				& Webapplikationen tilgås på både computer og telefon.
				& Applikationen tilpasser sig til den forbundne enhed.
				& Godkendt. \\ \hline

			3.2 & Load time skal være under 0.5 sekunder. 
				& Webapplikationens URL tilgås.
				& Siden er loadet inden 0.5 sekunder er gået.
				& Godkendt. \\ \hline
			
			3.3 & Bruger laver \newline flyveopsætning ved hjælp af kort.
				& Der startes en ny flyveopsætning og GPS koordinater vælges ud fra et kort.
				& Ved hjælp af kortet findes GPS koordinater.
				& Godkendt. \\ \hline				
		\end{tabular}
	\label{tab:krav_1}


\vspace{1cm}
    
    \begin{tabular}{|p{1. cm}|p{3.2 cm}|p{3.2 cm}|p{3.2 cm}|p{2.2 cm}|}
			\hline
			\multicolumn{5}{|l|}{\textbf{{\large Krav til drone}}}\\ \hline
			\textbf{Krav nr.} & \textbf{Krav} & \textbf{Test} & \textbf{Forventet \newline resultat} & 			
			\textbf{Resultat} \\ \hline
			
			4.1 & Skal forsynes fra batteri.
				& Batteri tilsluttes drone.
				& Lyde fra ESC'er indikerer at drone er klar til flyvning.
				& Godkendt. \\ \hline

			4.2 & Batterilevetiden skal minimum være 15 minutter.
				& Der laves teoretiske beregninger.
				& Beregnet levertid er over 15 minutter.
				& Godkendt. \\ \hline
			
			4.3 & Flyvehastigheden skal minimum være 2$\frac{m}{s}$.
				& Det noteres hvor lang tid dronen bruger på at flyve 10 meter.
				& Flyvehastigheden er minimum 2$\frac{m}{s}$.
				& Ikke \newline godkendt.\\ \hline		
				
			4.4 & Flyvehøjde kan reguleres i følgende 3 intervaller: 1-1.5m, 1.5-2.0m og 2-2.5m.
				& I flyveopsætning sættes flyvehøjde mellem 1-1.5 meter.
				& Drone flyver i den ønskede højde.
				& Ikke \newline godkendt.\\ \hline	
				
			4.5 & Højde der tages billeder fra, kan reguleres mellem 1 og 2,5 meter.
				& Den ønskede højde sendes til dronen.
				& Drone tager billeder i den ønskede højde.
				& Ikke \newline godkendt.\\ \hline	
		\end{tabular}
	\label{tab:krav_1}
	%\caption{Generelle krav}

\vspace{1cm}
   
    \begin{tabular}{|p{1. cm}|p{3.2 cm}|p{3.2 cm}|p{3.2 cm}|p{2.2 cm}|}
			\hline
			\multicolumn{5}{|l|}{\textbf{{\large Krav til opsamling af data}}}\\ \hline
			\textbf{Krav nr.} & \textbf{Krav} & \textbf{Test} & \textbf{Forventet \newline resultat} & 			
			\textbf{Resultat} \\ \hline
			
			5.1 & Tiden mellem et billede tages og til det er tilgængeligt på webapplikation skal maksimalt være 5 sekunder.
				& Tiden det tager at sende et billede måles.
				& Billedet tager maksimalt 5 sekunder om at nå til server.
				& Ikke \newline godkendt.\\ \hline			
			
			5.2 & Gyldig højdemåling ligger i intervallet 0,5 til 4,5 meter.
				& Drone flyttes højde sættes først til mindre end 0,5 meter. Derefter placeres den i en højden 0.5-4.5 meter. Til sidst sættes højde over intervallet. 
				& Drone går i fejlmode \#2 udenfor interval på 0,5-4,5 meter.
				& Godkendt. \\ \hline

			5.3 & GPS skal angive koordinat indenfor $\pm$ 2,5 meter. 
				& Drone flyttes rundt og GPS lokationen findes og verificeres.
				& GPS koordinater er angivet indenfor intervallet. 
				& Ikke \newline godkendt. \\ \hline		
		\end{tabular}
	\label{tab:krav_1}				
		