%%%%%% Kravspecifikation %%%%%%
%\newpage
\chapter{Funktionelle krav}

\label{sec:funkKrav}
De funktionelle krav beskrives via brugsscenarier, også kaldet use cases. Indledningsvis beskrives systemets aktører, og senere i afsnittet beskrives hvordan systemet fungerer ud fra interaktion mellem aktører og system.

\subsection{Aktør diagram}
%Følgende afsnit beskriver aktørerne i systemet.
Nedenstående figur viser hvilke aktører der interagerer med systemet.

\begin{figure}[H]
\centering
\includegraphics[width=1\textwidth]{Billeder/Aktor_diagram.png}
\caption{Aktør diagram}
\label{fig:ATD}
\end{figure}

\subsection{Aktørbeskrivelser}
Aktørbeskrivelsen skitserer systemets aktører samt hvilken rolle de spiller for systemet.


\begin{table}[H]
\begin{tabular}{|l|p{13.25cm}|} \hline

Navn					& Bruger. 	\\\hline
Type					& Primær.	\\\hline
Beskrivelse				& Bruger er den eneste person der interagerer med systemet.\\
						& Via webapplikation indstiller bruger flyveopsætning for nye flyvninger, \\ 
						& samt undersøge billeder og flyveruter fra tidligere flyvninger.\\\hline
						
\end{tabular}
\caption{Aktørbeskrivelse, Bruger}
\label{tab:AB1}
\end{table}


\begin{table}[H]
\begin{tabular}{|l|p{13.25cm}|}
\hline
Navn					& GPS satellitter. 	\\\hline
Type					& Sekundær.	\\\hline
Beskrivelse				& GPS satellitterne bruges når dronen lokaliserer sin position.\\\hline

\end{tabular}
\caption{Aktørbeskrivelse, GPS satellitter}
\label{tab:AB1}
\end{table}			

\newpage 

\subsection{Use case diagram}
\label{subsec:useCaseDiagram}
Figur \ref{fig:UCD} viser de identificerede use cases.
\vspace{-10pt}
%Usecase diagram indføres i følgende 5 linjer, derefter starter UC 1
\begin{figure}[H]
\includegraphics[width=1\textwidth]{Billeder/Use_case_diagram.png}
\vspace{-30pt}
\caption{Use case diagram}
\label{fig:UCD}
\end{figure}




\newpage
\subsection{Udviklingsforløb}

For at gøre udviklingsforløbet mere overskueligt gøres brug af iterationer.
De iterationer der ligger tidligst i udviklingsforløbet er mest essentielle for systemet, 
mens de iterationer der ligger senere i forløbet prioriteres knap så højt.
Udviklingsforløbet i dette projekt er planlagt til at forløbe via 4 iterationer. Nedenfor beskrives hvad der skal laves i de forskellige iterationer. \\

\textbf{Iteration 1:} \\
I denne iteration tages hånd om systemet mest grundlæggende funktionalitet. Batteri, ESC'er, motorer og ultralyds sensorer tilsluttes dronen. Desuden gøres dronen i stand til at kunne oprette forbindelse til internettet via 3G-shield. Når iteration 1 er færdig skal UC\#1 kunne gennemføres.  

\textbf{Iteration 2:} \\
I iteration 2 er hovedformålet at få styr på kommunikationen mellem webapplikation og drone. Det skal være muligt for bruger både at oprette og sende flyveopsætning til dronen. 
Ydermere skal drone kunne finde egen GPS position, flyvehøjde og orientering. Ud fra viden om egen position, flyvehøjde og orientering skal dronen kunne flyve til de lokationer bruger ønsker. Når iteration 2 er færdig skal UC\#2 og UC\#3 kunne gennemføres.  

\textbf{Iteration 3:}  \\
I iteration 3 er det primære fokus billeder. Der skal monteres kamera på dronen, så den kan tage billeder ved de lokationer som bruger har defineret. Alle billeder taget under flyvning sendes via mobilnet fra dronen til webapplikation og gøres tilgængelige for bruger. Når iteration 3 er færdig skal UC\#4 og UC\#5 kunne gennemføres.

\textbf{Iteration 4:} \\
I iteration 4 ønskes det at udvikle anti kollision til dronen. 
Inden tilføjelsen af anti kollision kan dronen udelukkende flyve i lukkede områder uden forhindringer. Men tilføjelse af anti kollision vil muliggøre flyvning i normale område forhindringer. Når iteration 3 er færdig skal UC\#6 kunne gennemføres  

\newpage
\section{Use case beskrivelse}
\label{subsec:useCaseBeskrivelse}
I afsnittet vises fully dressed use case beskrivelser af de use cases vist i på figur \ref{subsec:useCaseDiagram}. \newline 

\subsection*{UC1 - Start quadrocopter}

\begin{table}[H]
\begin{tabular}{|l|p{10cm}|}
\hline

Goal	 								& Quadrocopter er tændt og har forbindelse til webapplikation. \\\hline
Initiation 							& Bruger. \\\hline
No. of concurrent occurrence’s		& 1. \\\hline
Stakeholders	and Interests			& Bruger (primær) 
										\begin{itemize}
											\item Bruger tænder quadrocopter.
										\end{itemize} \\\hline
Precondition						& Ingen. \\\hline
Postcondition						& Quadrocopter er klar til at modtage flyveinstruktioner fra webapplikation. \\\hline
Main success scenario				&
 
									\renewcommand{\labelenumi}{\arabic{enumi}.}
									\renewcommand{\labelenumii}{\Roman{enumii}:}

									\begin{enumerate}[topsep=0.0cm, leftmargin=0.5cm]
										\item Bruger tænder quadrocopter.
										\item Quadrocopter initialiseres.
										\item Forbindelse fra quadrocopter til webapplikation oprettes.
											\begin{enumerate}[partopsep=4cm, topsep=0cm, leftmargin=1cm]
												\item Forbindelse kan ikke oprettes.
											\end{enumerate}
										
									\end{enumerate} \\\hline	

Extensions							& 

									\renewcommand{\labelenumi}{\Roman{enumi}:}
									\renewcommand{\labelenumii}{\alph{enumii})}
									\begin{enumerate}[topsep=0.0cm,leftmargin=0.5cm]
										\item Forbindelse kan ikke oprettes.
											\begin{enumerate}[topsep=0cm, leftmargin=1cm]
												\item Systemet indikerer at der ikke er forbindelse mellem quadrocopter og webapplikation.
											\end{enumerate}
									\end{enumerate} \\\hline	

\end{tabular}
\caption{Use Case 1}
\label{tab:UC1}
\end{table}
\subsection*{UC2 - Ny flyveopsætning}

\begin{table}[H]
\begin{tabular}{|l|p{10cm}|}
\hline

Mål	 							& Ny flyveopsætning er oprettet og sendt til quadrocopter. \\\hline
Initiering							& Bruger. \\\hline
Aktører og interesserede parter			& Bruger (primær) 
										\begin{itemize}
											\item Bruger ønsker at få adgang til webapplikation.
											\item Fra webapplikation indstiller bruger flyveopsætning.
										\end{itemize}
									  Webapplikation (sekundær)
										\begin{itemize}
											\item Sender flyveopsætning til quadropcopter.
										\end{itemize} \\\hline
Startbetingelser						& Bruger er oprettet i systemet og UC\#1 er succesfuld gennemført  \\\hline
Slutbetingelser						& Ny flyveopsætning er sendt til quadrocopter. \\\hline
Hovedforløb				&
 
									\renewcommand{\labelenumi}{\arabic{enumi}.}
									\renewcommand{\labelenumii}{\Roman{enumii}:}

									\begin{enumerate}[topsep=0.0cm, leftmargin=0.5cm]
										\item Bruger logger på webapplikation.
										\begin{enumerate}[partopsep=4cm, topsep=0cm, leftmargin=1cm]
												\item Fejl i login.
										\end{enumerate}
										\item Webapplikations forside vises.
										\item Fra forsiden navigerer bruger til flyveopsætning.
										\item Bruger laver en ny flyveopsætning.
										\begin{enumerate}[partopsep=4cm, topsep=0cm, leftmargin=1cm]
												\item Der laves ikke ny flyveopsætning.
										\end{enumerate}
										\item Bruger vælger at sende flyveopsætning til quadrocopter.
									\end{enumerate} \\\hline	

Undtagelser							& 

									\renewcommand{\labelenumi}{\Roman{enumi}:}
									\renewcommand{\labelenumii}{\alph{enumii})}
									\begin{enumerate}[topsep=0.0cm,leftmargin=0.5cm]
										\item Fejl i log-in.
											\begin{enumerate}[topsep=0cm, leftmargin=1cm]
												\item Bruger føres tilbage til login.
											\end{enumerate}
										\vspace{0.4cm}
										\item Der laves ikke ny flyveopsætning.
											\begin{enumerate}[topsep=0cm, leftmargin=1cm]
												\item Gemt flyveopsætning benyttes.
											\end{enumerate}
									\end{enumerate} \\\hline
										

\end{tabular}
\caption{Use Case 2}
\label{tab:UC2}
\end{table}
\subsection*{UC3 - Flyv til position}

\begin{table}[H]
\begin{tabular}{| p{3cm}| p{11.5cm}|}
\hline


Mål	 							& Drone flyver til ønsket position. \\\hline
Initiering 							& UC\#2 eller UC\#4. \\\hline
Aktører og \newline stakeholders			& Bruger (stakeholder) 

										\begin{itemize}
											\item Bruger ønsker at drone flyver som angivet i flyveopsætning.
										\end{itemize} \\  
										
										& GPS satellitter (sekundær aktør) 

										\begin{itemize}
											\item Dronen opdaterer egen GPS position vha. GPS satellitterne.
										\end{itemize} \\ \hline
Startbetingelser							& UC\#1 og UC\#2 er succesfuld gennemført. \\\hline
Slutbetingelser						& Position er nået. \\\hline
Hovedforløb				&
 
									\renewcommand{\labelenumi}{\arabic{enumi}.}
									\renewcommand{\labelenumii}{\Roman{enumii}:}

									\begin{enumerate}[topsep=0.0cm, leftmargin=0.5cm]
										\item Drone henter flyveopsætning fra server.
											\begin{enumerate}[a:,partopsep=4cm, topsep=0cm, leftmargin=1cm]
												\item Dronen henter ikke flyveopsætning.
											\end{enumerate}
										\item Nuværende position opdateres.
											\begin{enumerate}[b:,partopsep=4cm, topsep=0cm, leftmargin=1cm]
												\item Ugyldig GPS koordinat.
											\end{enumerate}
										\item Flyvehøjde tilpasses.
											\begin{enumerate}[c:,partopsep=4cm, topsep=0cm, leftmargin=1cm]
												\item Ugyldig flyvehøjde.
											\end{enumerate}
										\item Flyveorientering tilpasses.
										\item Drone flyver mod ønsket position.
										\item Ønsket position er nået.
										\begin{enumerate}[d:,partopsep=4cm, topsep=0cm, leftmargin=1cm]
												\item Ønsket position er ikke nået.
											\end{enumerate}
									\end{enumerate} \\\hline	

Undtagelser							& 

									\renewcommand{\labelenumi}{\Roman{enumi}:}
									\renewcommand{\labelenumii}{\alph{enumii})}
									\begin{enumerate}[a:,topsep=0.0cm,leftmargin=0.5cm]
										\item Dronen henter ikke flyveopsætning.
											\begin{itemize}[topsep=0cm, leftmargin=1cm]
												\item Flyvning med en anden flyveopsætning er aktiv.
											\end{itemize}
										\item Ugyldig GPS koordinat.
											\begin{itemize}[topsep=0cm, leftmargin=1cm]
												\item Drone går i fejlmode  \#1.
											\end{itemize}
										\item Ugyldig flyvehøjde.
											\begin{itemize}[topsep=0cm, leftmargin=1cm]
												\item Drone går i fejlmode \#2.
											\end{itemize}
										\item Ønsket position er ikke nået.
											\begin{itemize}[topsep=0cm, leftmargin=1cm]
												\item UC \#3 genstartes.
											\end{itemize}
									\end{enumerate} \\\hline	

\end{tabular}
\caption{Use Case 3}
\label{tab:UC3}
\end{table}
\subsection*{UC4 - Billede af position}

\begin{table}[H]
\begin{tabular}{| p{3cm}| p{11.5cm}|}
\hline

Mål	 							& Drone tager et billede af nuværende position som sendes til webapplikation. Fra webapplikation kan bruger inspicere og acceptere billedet. \\\hline
Initiering 							& UC\#3. \\\hline
Aktører og interesserede			& Bruger (primær) 
										\begin{itemize}
											\item Kan inspicere og acceptere billede.
										\end{itemize} 
									  Webapplikation (sekundær)
										\begin{itemize}
											\item Modtager billede fra drone.
											\item Viser bruger billede der skal accepteres.
										\end{itemize} \\\hline
Startbetingelser							& UC\#1, UC\#2 og UC\#3 er succesfuld gennemført. \\\hline
Slutbetingelser						& 	\begin{itemize}
											\item Bruger kan tilgå billede via webapplikation.
											\item Drone flyver til næste GPS-position eller udgangsposition.
										\end{itemize} \\\hline
Hovedforløb				&
 
									\renewcommand{\labelenumi}{\arabic{enumi}.}
									\renewcommand{\labelenumii}{\Roman{enumii}:}

									\begin{enumerate}[topsep=0.0cm, leftmargin=0.5cm]
										\item Drone tager et billede af nuværende position.
										\item Billedet sendes til webapplikation.
										\item Bruger giver accept af billedet via webapplikation.
											\begin{enumerate}[partopsep=4cm, topsep=0cm, leftmargin=1cm]
												\item Bruger accepterer ikke billede.
												\item Bruger svarer ikke inden for tidsgrænsen.
											\end{enumerate}
										\item Drone sendes information om næste lokation.
									\end{enumerate} \\\hline	

Undtagelser							& 

									\renewcommand{\labelenumi}{\Roman{enumi}:}
									\renewcommand{\labelenumii}{\alph{enumii})}
									\begin{enumerate}[topsep=0.0cm,leftmargin=0.5cm]
										\item Bruger accepterer ikke billede.
											\begin{enumerate}[topsep=0cm, leftmargin=1cm]
												\item Drone instrueres til at ændre højde, orientering eller position. Derefter genstartes UC4.
											\end{enumerate}
										\item Bruger svarer ikke inden for tidsgrænsen.
											\begin{enumerate}[topsep=0cm, leftmargin=1cm]
												\item Drone får automatisk tildelt accept.
											\end{enumerate}
									\end{enumerate} \\\hline	

\end{tabular}
\caption{Use Case 4}
\label{tab:UC4}
\end{table}
\subsection*{UC5 - Vis tidligere flyvning}

\begin{table}[H]
\begin{tabular}{|l|p{10cm}|}
\hline

Mål	 								& Bruger tilgår webapplikation hvor tidligere flyveruter og tilhørende billeder forefindes. \\\hline
Initiering 							& Bruger. \\\hline
Aktører og interesserede parter			& Bruger (primær) 
										\begin{itemize}
											\item Ønsker at få adgang til webapplikation. 
											\item Fra webapplikation kan bruger undersøge flyveruter og billeder fra tidligere flyvninger.
										\end{itemize} 
									  Webapplikation (sekundær)
										\begin{itemize}
											\item Giver bruger adgang til flyveruter og billeder.
										\end{itemize} \\\hline
Startbetingelser							& Bruger er oprettet i systemet. \\\hline
Slutbetingelser						& Ingen. \\\hline
Hovedforløb				&
 
									\renewcommand{\labelenumi}{\arabic{enumi}.}
									\renewcommand{\labelenumii}{\Roman{enumii}:}

									\begin{enumerate}[topsep=0.0cm, leftmargin=0.5cm]
										\item Bruger logger på webapplikation.
										\begin{enumerate}[partopsep=4cm, topsep=0cm, leftmargin=1cm]
												\item Fejl i log-in.
										\end{enumerate}
										\item Efter succesfuld log-in vises webapplikations forside.
										\item På forsiden vælges historik.
										\item Bruger vælger tidligere flyvning.	
										\item Flyverute samt billeder fra valgte flyvning vises.
									\end{enumerate} \\\hline	

Undtagelser							& 

									\renewcommand{\labelenumi}{\Roman{enumi}:}
									\renewcommand{\labelenumii}{\alph{enumii})}
									\begin{enumerate}[topsep=0.0cm,leftmargin=0.5cm]
										\item Fejl i log-in.
											\begin{enumerate}[topsep=0cm, leftmargin=1cm]
												\item Bruger bliver ført tilbage til log-in skærm.
											\end{enumerate}
									\end{enumerate} \\\hline	

\end{tabular}
\caption{Use Case 5}
\label{tab:UC5}
\end{table}
\subsection*{UC6 - Anti kollision}

\begin{table}[H]
\begin{tabular}{| p{3cm}| p{11.5cm}|}
\hline

Mål	 								& Drone kan undvige forhindringer vha. anti kollision \\\hline
Initiering 							& Ingen - altid aktiv. \\\hline
Aktører og \newline interesserede		& Bruger (primær) 
										\begin{itemize}
											\item Ønsker flyvning uden kollision.
										\end{itemize} \\\hline
Startbetingelser							& UC\#3 er igangværende. \\\hline
Slutbetingelser						& Drone har undveget en kollision. \\\hline

Hovedforløb				&
 
									\renewcommand{\labelenumi}{\arabic{enumi}.}
									\renewcommand{\labelenumii}{\Roman{enumii}:}

									\begin{enumerate}[topsep=0.0cm, leftmargin=0.5cm]

										\item Anti kollision detekterer en forhindring.
										\item Undvigningsmanøvre udføres.
											\begin{enumerate}[topsep=0cm, leftmargin=1cm]
												\item Forhindring kan ikke undviges.
											\end{enumerate}
									\end{enumerate} \\\hline	

Undtagelser							& 

									\renewcommand{\labelenumi}{\Roman{enumi}:}
									\renewcommand{\labelenumii}{\alph{enumii})}
									\begin{enumerate}[topsep=0.0cm,leftmargin=0.5cm]
										\item Forhindring kan ikke undviges.
											\begin{enumerate}[topsep=0cm, leftmargin=1cm]
												\item Drone går i fejlmode.
											\end{enumerate}
									\end{enumerate} \\\hline	

\end{tabular}
\caption{Use Case 6}
\label{tab:UC6}
\end{table}

\newpage
\chapter{Ikke-funktionelle krav}
\label{sec:ikkeFunkKrav}
De ikke-funktionelle krav indeholder specifikke krav som timings, afstande og lydniveauer.
Ikke-funktionelle inddeles i følgende 3 grupper: Generelle krav, krav til webapplikation og krav til drone.\newline


\begin{enumerate}
\item \textbf{Generelle krav}
	\begin{enumerate}[label*=\arabic*.]
	\item Kommunikation mellem drone og webapplikation skal foregå trådløst.
	\item Trådløs kommunikation benytter 3G protokol eller ældre. 
	\item Højdemåler skal måle højde $\pm$ 10 cm.\\
	\end{enumerate}

\item \textbf{Krav til server}
	\begin{enumerate}[label*=\arabic*.]
	\item Indholder database med billeder og flyveruter.
	\item Indholder database med brugere.\\
	\end{enumerate}	

\item \textbf{Krav til webapplikation}
	\begin{enumerate}[label*=\arabic*.]
	\item Webapplikation skal kunne tilgås via både computere og telefoner.
	\item Load time skal være under 0.5 sekunder.
	\item Bruger laver flyveopsætning ved hjælp af kort.\\
	\end{enumerate}	
	

\item \textbf{Krav til dronen}
	\begin{enumerate}[label*=\arabic*.]
	\item Skal forsynes fra batteri.
	\item Batterilevetiden skal minimum være 15 minutter.
	\item Flyvehastigheden skal minimum være 2$\frac{m}{s}$.
	\item Flyvehøjde kan reguleres i følgende 3 intervaller: 1-1.5m, 1.5-2m og 2-2.5m.
	\item Højde der tages billeder fra kan reguleres mellem 1 og 2,5 meter.\\
	
	\end{enumerate}
	
\item \textbf{Krav til opsamling af data}
	\begin{enumerate} [label*=\arabic*.]
	\item Tiden mellem et billede tages og til det er tilgængeligt på webapplikation skal maksimalt være 5 sekunder.	
	\item Gyldig højdemåling ligger i intervallet 0,5 til 4,5 meter. 
	\item GPS skal angive koordinat indenfor $\pm$ 2,5 meter.\\
	\end{enumerate}
\end{enumerate}