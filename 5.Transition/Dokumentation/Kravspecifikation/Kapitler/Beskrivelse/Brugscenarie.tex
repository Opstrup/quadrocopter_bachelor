\section{Brugsscenarie}

Et typisk brugsscenarie for den autonome overvågningsdrone vil være overvågning af et privat område eller virksomhed. Dronen kan nemt og omkostningsfrit overvåge et større område, hvilket betyder at brug af dronen vil spare menneskelige ressourcer. 

Dronen kræver to ting for at fungere. Dels skal den regelmæssigt have udskiftet eller opladet sit batteri, og desuden skal den have besked om hvad og hvor den skal overvåge.

Dronen opsættes via en webapplikation. Fra webapplikationen kan bruger indtaste hvor højt dronen skal flyve, hvilken højde der skal tages overvågningsbilleder fra og hvor der skal tages overvågningsbilleder.



\section{Prioritering}

\begin{table}[H]
	\centering
		\begin{tabular}{|l|l|p{9 cm}|} 
		\hline
			Område & Prioritering & Kommentar \\ \hline
			Sikkerhed 		& 5 	& Sikkerheden prioriteres højst, idet dronens propeller er farlige og kan skade mennesker og dyr.   \\ \hline
			
			Pålidelighed 	& 4 	& Hele systemet skal være meget pålidelig, da dronen under flyvning aldrig må fejle. Da fejl eller svigt kan udsætte mennesker og dyr for fare.  \\ \hline
			
			Pris 			& 2 	& Prisen er mindre vigtig, da der er stor interesse for udviklingen af denne type overvågning.    \\ \hline
			
			Brugervenlighed & 3 	& Systemet skal ikke kunne betjenes af alle, derfor er brugervenligheden ikke den vigtigste prioritering. \\ \hline
		\end{tabular}
	\caption{Prioriteringsliste}
\end{table}
