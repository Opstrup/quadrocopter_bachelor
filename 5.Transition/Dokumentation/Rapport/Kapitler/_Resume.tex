 %%%%%% Resumé %%%%%%
\chapter{Resumé og Abstract}
\label{chap:resume}


\section*{Abstract}

The purpose of this bachelor project is to develop an autonomous surveillance drone that free of charge and easily can monitor a given area. The idea is to develop an alternative to expensive and resource demanding surveillance performed by humans. This project is designed and developed on Aarhus school of engineering. 

A quadrocopter is being built into a drone which is used for surveillance. The user will be able to create a set of coordinates on a webapplication, the drone will then autonomous fly to the given coordinates and monitor that area. 
The drone consists of a Aeroquad ARF quadrocopter, a GPS, heightsensors and a mobile communication module. 
The user can view previous flights and create a new flightsetup through the webapplication.

During the project, a full functioning server, the coomunicationlink and exchange of data between server and the drone was successfully implemented. In addition data from the GPS, compass and heightsensors was collected and processed. Furthermore the main part of the webapplication and the autonomous flight system was implemented.




\section*{Resumé}

Formålet med dette bachelorprojekt er at udvikle en overvågningsdrone der nemt og omkostningsfrit kan overvåge et område. Dronen udvikles for at skabe et alternativ til dyr og ressourcekrævende overvågning udført af mennesker.
Projektet er designet og udviklet på Ingeniørhøjskolen Aarhus Universitet.

I projektet omdannes en fjernstyret quadrocopter til en drone, som skal bruges til overvågning. Ud fra brugers anvisninger skal dronen autonomt kunne flyve til GPS positioner og tage overvågningsbilleder.  

Den udviklede drone består af en Aeroquad ARF quadrocopter med påmonteret GPS, højdemåler og mobilt kommunikationsmodul. Systemets bruger indstiller droneflyvning samt monitorerer data fra tidligere flyvninger via en webapplikation.  

I projektforløbet lykkedes det at implementere en fuldt fungerende server, samt kommunikationslink og udveksling af data mellem server og drone. Desuden blev opsamling af data fra GPS, kompas og højdemåler fuldt ud implementeret. Det lykkedes endvidere at implementere hovedparten af webapplikationen og dronens autonome flyvning.





