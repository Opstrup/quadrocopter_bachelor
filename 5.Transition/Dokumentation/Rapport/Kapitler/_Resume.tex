 %%%%%% Resumé %%%%%%
\chapter{Resumé og Abstract}
\label{chap:resume}


\section*{Abstract}


\section*{Resumé}

Formålet med dette bachelorprojekt er at udvikle en overvågningsdrone der nemt og omkostningsfrit kan overvåge et område. Dronen udvikles for at skabe et alternativ til dyr og ressourcekrævende overvågning udført af mennesker.
Projektet er designet og udviklet på Ingeniørhøjskolen Aarhus Universitet.

I projektet omdannes en fjernstyret quadrocopter til en drone, som skal bruges til overvågning. Ud fra brugers anvisninger skal drone autonomt kunne flyve til GPS positioner og tage overvågningsbilleder.  

Den udviklede drone består af en Aeroquad ARF quadrocopter med påmonteret GPS, højdemåler og mobilt kommunikationsmodul. Systemets bruger indstille droneflyvning samt monitorerer data fra tidligere flyvninger via en webapplikation.  

I projektforløbet lykkedes det at implementere en fuldt fungerende server, samt kommunikationslink og udveksling af data mellem server og drone. Desuden blev opsamling af data fra GPS, kompas og højdemåler fuldt ud implementeret. Det lykkedes endvidere at implementere hovedparten af webapplikationen og dronens autonome flyvning.





