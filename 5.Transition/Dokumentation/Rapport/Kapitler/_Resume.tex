 %%%%%% Resumé %%%%%%
\chapter{Resumé og Abstract}
\label{chap:resume}


\section*{Abstract}


\section*{Resumé}

Denne rapport beskriver et bachelorprojekt fra Ingeniørhøjskolen Aarhus Universitet.

Der er i dag et stigende fokus på overvågning og sikkerhed. Formålet med bachelorprojektet var at udvikle en overvågningsdrone, der nemt og omkostningsfrit kunne overvåge et område. Dronen blev udviklet for skabe et alternativ til dyr og ressourcekrævende overvågning udført af mennesker. 

Via en webapplikation har systemets bruger mulighed for at oprette nye flyveopsætninger samt monitorerer flyveruter og billeder fra tidlige flyvninger. Når bruger opretter en ny flyveopsætning vælges en række GPS positioner som dronen skal flyve til. Desuden vælges flyvehøjde og hvorvidt der skal tages billeder ved de valgte GPS positioner. Hvis dronen tager billeder under flyvning, overføres billederne til en server.

I projektforløbet lykkedes det at implementere en fuldt fungerende server, samt kommunikationslink og udveksling af data mellem server og drone. Desuden blev opsamling af data fra GPS, kompas og højdemåler fuldt ud implementeret. Det lykkedes endvidere at implementere hovedparten af webapplikationen og den autonome flyvning. 





