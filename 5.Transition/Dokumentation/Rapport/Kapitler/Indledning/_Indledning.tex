%%%%%% Indledning %%%%%%
\chapter{Indledning}
\label{chap:indledning}

I den moderne verden findes der mange forskellige former for overvågning, og overvågning foregår overalt. Primært bruges overvågning til at skabe tryghed og forebygge mod kriminalitet, hvilket de fleste mennesker er positivt stemt overfor. Men pga. en stigning i brugen af overvågning, bruges der gradvist flere og flere ressourcer på området. 
I projektet undersøges det, hvorvidt det er muligt at udvikle en autonom overvågningsdrone. Det er gruppens mål at udvikle en drone der ud fra administrators anvisninger kan overvåge og tage billeder af et defineret område. Dronen udvikles for at mindske mængden af menneskelige ressourcer der bruges på overvågning.

Formålet med projektet er at konstruere en drone, som autonomt kan overvåge et givet område. Dronen gøres i stand til at kommunikerer med en server, samt orienterer sig om egen GPS position, flyvehøjde og flyveretning. Gennem kommunikation med server henter drone en flyveopsætning og flyver på egen hånd ud og overvåger det definerede område.
Projektet er planlagt, designet og konstrueret hen over en periode på 4 måneder.

Projektrapporten er sat op efter en standard, hvor systemet indledningsvis beskrives på et abstrakt niveau. Herefter beskrives krav til produktet, og der gåes mere analytisk og teknisk til værks. Rapporten afsluttes af med resultater, diskussion af resultater samt en konklusion.


