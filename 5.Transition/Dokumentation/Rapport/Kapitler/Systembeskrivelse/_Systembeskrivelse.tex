%%%%%% Systembeskrivelse %%%%%%
\chapter{Systembeskrivelse}
\label{chap:systembeskrivelse}

Track’n’Play er et musikanlæg, som, udover at kunne afspille musik, også automatisk justerer højtalerretning og lydstyrke. Justeringen sker for at forbedre brugerens lydoplevelse.

Systemet består af et lydanlæg, en medieafspiller, to højtalere på roterende platforme, en bærbar enhed samt en positionsbestemmer. 

\begin{figure}[h!]
  \centering
    \includegraphics[width=0.8\textwidth]{billeder/systembeskrivelse/trackplay opstilling.pdf}
    \caption{Track'n'Play opstilling}
\end{figure}


Anlægget modtager et lydsignal fra medieafspilleren, som den forstærker og afspiller via højtalerne.\newline
På den bærbare enhed kan brugeren se den aktuelle lydstyrke for anlægget, samt justere lydstyrken til et andet niveau. Den bærbare enhed bruges også til at måle lydstyrken i brugerens position, så brugeren altid oplever en ensartet lydstyrke selvom afstanden mellem anlægget og brugeren ændres. Datakommunikation mellem anlægget og den bærbare enhed foregår trådløst via en radiosender.\newline
Den bærbare enhed udsender, udover et radiosignal til anlægget, også en højfrekvent tone ud i rummet, som ikke kan høres af det menneskelige øre. Denne tone opfanges af positionsbestemmelsen og ud fra denne tone kan brugerens position i rummet beregnes.\newline
Når brugerens position er fastlagt, sendes den over til anlægget, der justerer højtalernes retning, så de peger imod brugeren. 