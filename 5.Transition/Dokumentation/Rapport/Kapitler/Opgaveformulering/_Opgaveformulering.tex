%%%%%% Opgaveformulering %%%%%%
\chapter{Opgaveformulering}
\label{chap:opgaveformulering}

Denne opgaveformulering er medtaget i rapporten for at give et overblik over hvilke krav, der på forhånd var givet. Der er endvidere udarbejdet en specifik projektformulering for projektet Track’n’Play.


\section{Krav til projekt}
\textbf{4. semester-projektet består blandt andet i at:}
\begin{enumerate}
\item Beskrive et udviklingsprojekt i en rapport– og dokumentationsdel over et valgt emne bestående af HW og SW. 
\item Beskrive, afgrænse, analysere, designe, implementere og teste det valgte projektemne.
\item Vurdere den opnåede løsnings stærke og svage sider i relation til det tiltænkte.
\end{enumerate}\medskip

\textbf{Krav til semesterprojektet}
Systemet kan vælges frit, forudsat følgende overordnede krav er opfyldt:
\begin{enumerate}
\item Systemet skal interagere med omverdenen ved hjælp af sensorer og aktuatorer.
\item Der skal anvendes relevante faglige elementer fra semestrets kurser.
\item Systemet skal omfatte pålidelig transmission af data mellem udvalgte enheder.
\item Systemet skal kunne interagere med en bruger.
\end{enumerate}\medskip 


\section{Projektformulering}
Der udvikles på baggrund af ovenstående et lydsystem, der automatisk roterer højtalernes retning, således de er orienteret mod brugerens position, samt automatisk justere lydstyrken, så lydtrykket hos brugeren er konstant, uanset dennes position. Systemet optimerer derved lydbilledet for brugeren.\newline
Brugeren skal bære rundt på en bærbar enhed, hvorfra aktuel ønsket lydstyrke kan aflæses, samt ændres. Den bærbare enhed skal også bruges til at bestemme det aktuelle lydtryk samt brugerens position.

