\chapter{Implementation}
Implementeringen af systemet tager udgangspunkt i de udarbejdede pakke-, sekvens- og klassediagrammer. De højst prioriterede use cases er blevet delvist implementerede. Dette afsnit beskriver hvordan div. delsystemer er udviklet overordnet.
For at danne et overblik over den forliggende opgave blev der i starten af hver iteration oprettet pakkediagrammer. Ud fra disse pakkediagrammer blev der udarbejdet sekvensdiagrammer, disse diagrammer lå så til grundlag for klassediagrammerne.

\section{Drone}
Dronen er udviklet i programmerings sproget C++. Software tilhørende dronen er opdelt i forskellige ansvars områder og disse områder er designet til at fungere med de hw-enheder de benytter. Dronen har bla. kode der aflæser højde sensor, GPS-lokation og HTTP-kommunikation, informationen fra disse sw-enheder bliver behandlet i en main controller fil på dronen og ud fra dette afgøres hvilke handling dronen skal udføre.

\section{Server}
Under iteration 1 blev der lagt meget arbejde på serveren, da den spiller en væsentlig rolle til kommunikationen imellen drone og webapplikation. Serveren er udviklet i programmerings sproget Python og med webframeworket Django[x] dækker det tilsammen over en SQLite database med et RESTful api. Serveren har en række API-endpoints som benyttes for til at data'en på serveren.

\section{Webapplikation}
Webapplikationen  er udviklet i programmerings sprogene HTML, CSS og JavaScript. Webapplikationen er udviklet med web frontend frameworket AngularJS[x], som er et meget benyttet framework fra Google. Webapplikationen benytter yderligere et google maps API, hvilket giver den funktionalitet for tilføjelse af et kort på sitet. 
Da AnjularJS er benyttet til udviklingen, er projektet også opdelt efter MVC-modellen[x]. Dette giver projektet en rigtig god struktur og opdeling af filer.