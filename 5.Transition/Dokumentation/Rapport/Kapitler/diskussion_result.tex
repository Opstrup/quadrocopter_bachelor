\section{Diskussion af resultater}

I dette afsnit beskrives og diskuteres de resultater gruppen opnåede og der laves en beskrivelse af de punkter der kunne være lavet på bedre, smartere eller anderledes vis. Diskussionen tager udgangspunkt i de opnåede resultater beskrevet i resultatafsnittet.

Accepttesten forløb stort set som planlagt indtil test case 4-6, som omhandlede opsamling af billeder, visning af billeder og tilføjelse af antikollision. Test case 4-6 blev ikke godkendt, da de kun blev designet og ikke implementeret. Da arbejdsprocessen var planlagt i iterationer, var de fejlede test cases ikke kritiske for det resterende system.\\

\textbf{Drone}\\
Drone kan under flyvning finde nuværende flyvehøjde, flyveretning og GPS position ved aflæsning af data fra ultralydssensor, kompas og GPS.  Ud fra det aflæste data tilpasses flyveindstillinger. Via det mobile 3G-netværk sender drone information om nuværende position til server og henter flyveopsætninger fra server.

Tilpasning af flyveindstillinger er implementeret i tre trin. Først tilpasses flyvehøjde, dernæst tilpasses flyveretning og når både flyvehøjde og flyveretning er tilpasset flyves der fremad. Hver del er implementeret og fungerer hver for sig selv, men samlet set fungerer det ikke optimalt pga. manglende optimering. Til justering af flyvehøjde og flyveorientering kunne der med fordel være gjort brug af PID regulering.  I implementerings fasen opstod problemer med timing, specielt når 3G/GPS modulet blev brugt. Ved at indføre brug af flertrådet main controller kunne disse problemer undgås. \\


\textbf{Webapplikation}\\
Før systemets bruger kan få lov at tilgå webapplikationens funktionalitet er login påkrævet. Efter succesfuld login hentes og fremvises data fra server.
Når bruger er logget ind kan der laves nye flyveopsætninger, men pga. manglede implementering kan webapplikationen ikke sende de nyoprettede flyveopsætninger til server.

Overordnet designvalg til webapplikation har fungeret tilfredsstillende. Dette har betydet at der undervejs i projektet ikke er lavet væsentlige ændringer i design og brug af udviklingsværktøjer. Webapplikationen er udviklet ved brug AngularJS, CSS3 og HTML5. Server er udviklet med en kombination af Django og Django-REST-framework, på samme vis kunne webapplikationen være blevet udviklet. Men server og webapplikation er bevidst udviklet adskilt og med forskellige værktøjer. Dette er gjort for at bevise at der er lav kobling mellem server og webapplikation, og for at undgå problemer med syntaks. 

\newpage

\textbf{Server}\\
Det er lykkedes at implementere en fuld funktionel server, der fungerer som bindeled mellem drone og webapplikation. Serveren er implementeret som en passiv enhed, der er ansvarlig for håndtering af systemets data og tilgås via HTTP protokollen. 

Billeder i serverens database er koblet op til events, de kunne med fordel i stedet være koblet til waypoints. Dette havde givet en bedre kobling med GPS koordinater, og havde overordnet set forbedret database designet. 
Udover dette har de overordnet server design valg fungeret tilfredsstillende. Dette har betydet at der er blevet holdt fast i de værktøjer og beslutninger, der blev truffet i starten af projektet. Eksempelvis indeholder server en SQLite database, som til projektet har fungeret fint. Der kunne dog i stedet være gjort brug af en MongoDB database, der er en databasetype der håndterer data som objekter.

\vspace{0.5cm}

\subsection{Færdigimplementering af manglende funktionalitet}

Som det fremgik af resultatafsnittet, er det ikke lykkedes at implementere systemets fulde funktionalitet. Derfor bør der inversteres tid i færdigudvikling af systemet før eventuel videreudvikling foretages. I dette afsnit beskrives hvordan det restende systemfunktionalitet kan implementeres. 

\textbf{Test case 4}  \\
Før test case 4 kan gennemføres succesfuldt, skal håndtering af kamera implementeres.
Sekvens- og klassediagrammer er designet, så det eneste der mangler er oprettelse af en \textit{camera} klasse. Klassen oprettes og tilføjes de metoder der er beskrevet i klassediagrammet. Desuden skal dronen være i stand til at sende billeder til server og tage imod accept af billeder fra systemets bruger. Dette gøres ved udvidelse af \textit{communication} klassen. 

\textbf{Test case 5} \\
For at test case 5 kan gennemføres skal logikken til archive view'et implementeres. 
Helt overordnet er målet, at visning af billeder og flyveruter skal implementeres, så billeder og flyveruter fra forskellige flyvninger præsenteres på korrekt vis. Dette gøres bla. ved brug af kalender funktionen i AngularJS. 

\textbf{Test case 6}\\
Test case 6 indeholder antikollision. Antikollision skal sikre at dronen ikke flyver ind i eventuelle forhindringer uden flyvning. For at implementere antikollisions funktionaliteten, kan der gøres brug af \textit{DistanceSensor} klassen. Det er muligt at lave objekter af \textit{DistanceSensor} klassen som kan bruges til altikollision. 

\newpage

\subsection{Videreudvikling}

I dette afsnit beskrives nogle af de muligheder der er for videreudvikling af systemet.  \\

\textbf{Optimering af højdemåling}\\
Den ultralydssensor der bruges til aflæsning af flyvehøjde kan maksimalt måle afstande på 4-5 meter og er knap så præcis, når dronen er i bevægelse. 
Til at optimere måling af flyvehøjden kan systemet udvides med endnu en teknologi der måler flyvehøjde. En løsning kan være at tilføje andre sensorer, der i fællesskab med ultralydssensoren bruges til måling af flyvehøjde. 
På flight control board'et befinder der sig et barometer, dette kan eksempelvis benyttes til at supplere ultralydssensoren.
Ved kombination af to teknologier, vil ultralydssensoren kunne anvendes når dronen skal lette eller lande og barometeret kan anvendes under flyvning. 
Måling af flyvehøjde med både ultralydssensor og barometer vil være en stabil løsning, da begge teknologier har stærke og svage sider, og de svage fjernes ved kombination af teknologierne.\\


\textbf{PID regulering}\\
Flyvehøjde og orientering er to parametre der konstant skal reguleres for at sikre dronen flyver på bedst mulig vis. Til tilpasning af flyvehøjde og orientering anvender dronen simpel regulering, der er bygget op omkring det data der aflæses fra højdemåler og kompas.
I den implementerede regulering tages der ikke forbehold for tidligere fejl, hvilket betyder reguleringen får flyvehøjde og orientering til at svinge meget. 
Et alternativ til den nuværende regulering kunne være implementering af PID regulering. Med PID regulering måles og sammenlignes nuværende fejl med den tidligere fejl og reguleringen indstilles på mere dynamisk og flydende vis. \\


\textbf{Flertrådet main controller} \\
Den nuværende main controller kan ikke håndtere flere tråde samtidig, hvilket betyder main controlleren kun kan afvikle en metode af gangen. Dette giver i nogle tilfælde problemer og betyder at kritiske metoder komme til at vente på mindre kritiske metoder færdigafvikles. Aflæsning af GPS position fra 3G/GPS modulet er et eksempel på en tidskrævende men ikke særlig kritisk metode. Men i det nuværende system har aflæsning af GPS position samme prioritet som tilpasning af flyvehøjde og orientering, hvilket kan skabe kritiske situationer. Ved at anvendelse af et flertrådet system, kan der indføres prioriteringsliste og afviklingen af forskellige metoder kan optimeres. Desuden vil implementering af en flertrådet main controller tillade, at aflæsning af GPS position vil kunne afvikles samtidig som tilpasning af flyvehøjde og orientering. \\

\textbf{Sikkerhed af webapplikation og server}\\
På nuværende tidspunkt er der ikke ingen sikkerhed i forbindelse med webapplikation og server. Derfor bør øget sikkerhed være et fokuspunkt for fremtidig videreudvikling af systemet. Eksempelvis kunne der benyttes tokens i kommunikationen mellem server og webapplikation og kryptering af brugernavne/passwords i databasen.
