\section{Udviklingsproces}
Projektforløbet er gennemført over en periode på 4 måneder, hvor der er blevet gjort brug af forskellige arbejdsmetoder og værktøjer. I afsnittet beskrives hvorledes projektet er gennemført, hvilke metoder der er anvendt samt hvilke udviklingsværktøjer der er benyttet.

\subsection{Gennemførelse}
Projektet er gennemført via fire iterationer, hvor hver iteration har indeholdt en design- og implementeringsfase. Nedenfor beskrives de fire iteration. Alle gruppens medlemmer har arbejdet tæt sammen under hele forløbet.

Alle gruppens medlemmer har arbejdet tæt sammen under hele forløbet. Specielt i udarbejdelsen af foranalyse, krav og indledende systemarkitektur blev der arbejdet i fællesskab. 
Da projektet bevægede sig over i detaljeret design og implementering, blev der gjort brug af fire iterationer. Detaljeret design og implementering er opdelt og udarbejdet af af enkelte gruppemedlemmer. Nedenfor beskrives hvad iterativ udvikling er og hvad der blev arbejdet med i de fire iterationer.


\subsubsection{Iterativt udvikling}
Stort set alle udviklingsprojekt forløber via vandfaldsmodel eller iterativt udvikling. 
Hvis vanfaldsmodellen benyttes gennemløber projektet en række faser, der hver for sig afsluttes, inden næste fase påbegyndes. Opbygning af vanfaldsmodellen kunne eksempelvis være: Krav, analyse, design, implementering og til sidst test
I et iterativt projektforløb gennemløber projektet i stedet en række faser, der hver for sig kan ses som miniudgaver af vandfaldsmodellen. Iterationer der ligger tidligt i et projektforløb vil ofte omhandle grundfunktionalitet, mens iterationer senere i projektforløbet bruges til at tilføje funktionalitet til systemet. 

Da systemarkitektur og desgin er opdelt i use cases, var det oplagt at opdele detaljeret design og implementering  i iterationer. Detaljeret design og implementering opdeles i fire iterationer, og nedenfor beskrives indholdet af de forskellige iterationer.  

\textbf{Iteration 1}\\
I den første iteration er fokus på systemets mest grundlæggende funktionalitet. 
Drone gøres i stand til at oprette forbindelse til server via 3G-shield'et.
Desuden tilsluttes batteri, ESC'er, motorer og ultralyds sensorer til drone. 
Målet med iterationen er at kunne gennemføre use case 1. 

\textbf{Iteration 2}\\
I iteration 2 er hovedformålet at få kommunikation mellem server og drone til at fungere. Bruger skal kunne oprette nye flyveopsætninger og gøre dem tilgængelig på server for dronen. Ydermere skal drone kunne finde egen GPS position, flyvehøjde og orientering. Ud fra viden om egen position, flyvehøjde og orientering skulle dronen kunne flyve til de forudbestemt lokationer af brugeren. Efter denne iteration skal use case 2 og 3 kunne gennemføres.

\textbf{Iteration 3}\\
I iteration 3 er hovedformål at tilføje billede håndtering. Der monteres kamera på drone, så der kan tages billeder under flyvning. Billeder taget under flyvning sendes via mobilnet fra drone til server og gøres tilgængelige for bruger. Målet med iteration 3 er at kunne gennemføre use case 4 og 5.

\textbf{Iteration 4}\\
Iteration 4's hovedformål var at udvikle anti kollision til dronen. Inden tilføjelsen af anti kollision kunne dronen udelukkende flyve i lukkede områder uden forhindringer. Tilføjelsen af anti kollision skal muliggøre flyvning i normale områder med forhindringer. Efter denne iteration skal alle use cases kunne gennemføres.

\newpage
\subsection{Metoder}

\subsubsection{RUP model}
\textbf{R}ational \textbf{U}nified \textbf{P}rocess er blevet anvendt til at holde den overordnet fremgangs metode til projektet.  

\begin{figure}[H]
	\centering
	\includegraphics[width=0.60\textwidth]{Billeder/Udviklingsproces/RUP}
	\caption{RUP model}
	\label{fig:rup}
\end{figure}

\textbf{Inception}\\
I denne fase blev de indledende dokumenter udarbejdet. Der blev lavet forundersøgelse og valg vedrørende hardware. En indkøbs liste blev også udarbejdet i denne fase. Krav til systemet blev også udarbejdet i denne fase samt en overordnet systemskitse. 

\textbf{Elaboration}\\
I denne fase blev domain modellen udarbejdet, der blev fortaget en use case analyse af systemets funktionalitet. Ved brug af use case analysen blev der også udarbejdet en iterations plan over den kommende udvikling af systemet. 

\textbf{Construction}\\
I denne fase var produktet i fokus. Udviklingen forgik skiftevis mellem implementering af nyt funktionalitet, test og dokumentation af færdiggjort funktionalitet.

\textbf{Transition}\\
I denne fase blev der gjort opsamling af dokumentationen og rapporten samt accepttest.

\newpage

\subsubsection{ASE modellen}
I kombination med RUP arbejdsmetoden og ASE modellen som ses på figur \ref{fig:dokument_udvikling} er dokumentationen opdelt i fire dokumenter, hvilket giver et godt overblik over de forskellige dokumenter tilhørende projektet. Software delen af dokumentationen er så yderligere opdelt efter modellen n + 1 som beskrevet .

\begin{figure}[H]
	\centering
	\includegraphics[width=1\textwidth]{Billeder/Udviklingsproces/ase_model}
	\caption{ASE modellen}
	\label{fig:dokument_udvikling}
\end{figure}

\subsubsection{Agile arbejdsmetoder}
Udviklingen og dokumentationen af produktet har også været meget præget af agile arbejdsmetoder. Metoder som scrum er blevet delvis anvendt i forløbet. Under construction fasen blev scrum meetings brugt og en samlet backlog blev benyttet til at bevare overblik over projektets fremgang.

\newpage

\subsubsection{N + 1 model}
Dokumentationen af dette projekt har fulgt dokumentations modelle N + 1. N + 1 modellen er en dokumentations model for software, modelle deler dokumentationen af software systemet op i forskellige view's. Til dette projekt er det blevet benyttet 5 + 1 modelle. 5 + 1 modellen indeholder use case view, logical view, process view, data view, deployment view og implementation view. De nævnte view's vil blive beskrevet mere i dybden nedenfor. 

\begin{figure}[H]
	\centering
	\includegraphics[width=0.85\textwidth]{Billeder/Udviklingsproces/n+1}
	\caption{n + 1 modellen}
	\label{fig:n+1}
\end{figure}

\textbf{Use Case View}\\
Dette view er benyttet på baggrund af opdelings muligheder af projektet. View'et giver også et mere håndgribeligt overblik over systemet, da fokus i view'et ligger på funktionalitet for brugeren af systemet. Under dette view i dokumentationen finde use case diagrammer, hvilket tydeliggøre sammenhæng imellem aktør, bruger og systemet.

\textbf{Logical View}\\
Dette view er benyttet til at beskrive de logiske blokke i systemet.

På baggrund af hver iteration i systemet er der udarbejdet dertilhørende pakke-, klasse- og sekvensdiagrammer. Dette giver et overblik over systemets funktionalitet på programmerings niveau.

\newpage

\textbf{Deployment View}\\
Dette view er benyttet til at beskrive det fysiske elementer i systemet. Her beskrives også hvor de forskellige pakker befinder sig i systemet. Hvilket protokoller som er anvendt som interface imellem de forskellige enheder er også beskrevet i dette view.

\textbf{Process View}\\
Dette view er benyttet til at beskrive hvilke sideløbende processer der befinder sig i systemet. I view'et findes også beskrivelse af kommunikations timing, da i nogle tilfælde i forbindelse med kommunikation er det vigtigt at kommunikationen bliver afviklet i en bestemt rækkefølge.

\textbf{Data View}\\
Dette view er benyttet til at beskrive databasen i systemet. Håndteringen af dataen og hvordan den tilgås er også beskrevet i dette afsnit. 

\textbf{Implementation View}\\
Dette view er benyttet til at beskrive de vigtigste elementer i systemet som ikke er blevet beskrevet i de andre views. I dette view er opsættelse af systemet til vider udvikling og hvordan systemet kompileres beskrevet.

\subsection{SW og HW diagrammer}
Til dokumentationen af projektet er der benyttet en række diagrammer, disse diagrammer vil blive beskrevet i dette afsnit.

\textbf{Use case}\\
Use cases og dertilhørende diagrammer er benyttet i forløbets indledende faser, til at overskueliggøre projektets omfang og opdele projektet i mindre dele. Use casene har fungeret som omdrejnings punkt i projektet, det er her funktionaliteten udspringer fra.

\textbf{Domainmodel}\\
I de indledende faser blev der også udarbejdet en domainmodel som blev brugt til at finde ansvars områder i projektet.

\textbf{SysML}\\
SysMLl diagrammerne er blevet benyttet til dokumentation af hardwaren til projektet. Disse diagrammer er blevet anvendt til at tydeliggøre, hvilke hardware blokke der høre til hvor og hvordan de kommunikere sammen.  

\textbf{UML}\\
UML diagrammerne er blevet  benyttet til dokumentation af softwaren. Disse diagrammer er blevet anvendt til at forklare hvordan systemet er opbygget, hvordan de forskellige delsystemer snakker sammen og hvordan funktionaliteten er opbygget.