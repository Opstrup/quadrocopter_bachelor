\chapter{Projektafgrænsning}

Ud fra kravspecifikationen og systemskitsen figur \ref{fig:Systemskitse} var der dannet en grundlæggende idé om hvordan systemet skulle se ud og fungere. Men for at blive i stand til at sætte realistiske mål for projektet, blev det besluttet at lave en afgrænsning af systemets funktionalitet. I dette afsnit beskrives nogle af de afgrænsninger gruppen har valgt at lave. \\


\textbf{Video live feed og billedekvalitet}\\
En af de første afgrænsninger der blev lavet, var afgrænsningen af kamera med høj billedkvalitet som kunne optage video. Planen om høj kvalitets billeder og live stream blev ændret, fordi upload-hastigheden på på 3G / 2G netværket er begrænset, og gruppen var bange for det kunne blive en flaskehals. Idéen om live feed blev droppet og i stedet blev det besluttet at gøre brug af et kamera med lav opløsning, som løbende skulle tage almindelig billeder.\\

\textbf{Tilgang til webapplikation}\\
I kravspecifikationen beskrives det, at ethvert device med internet adgang skal kunne benytte webapplikationen. Men afhængigt af hvilket device der benytter webapplikationen, kræves en ny opsætning. Derfor blev det besluttet kun at designe og implementere webapplikationen til således den kunne bruges via computere.\\

\textbf{Flyvehøjde og tilhørende sensorer}\\
Hvis dronen skulle flyve over længere distancer ville flyvehøjde over 5 meter helt klart være fordelagtigt. Primært ville det være en fordel at kunne flyve højere, fordi der færre forhindringer jo højere der flyves. Men der er bevidst valgt en teknologi, der maksimalt kan bruges til måling af afstande på 4-5 meter. Teknologien er valgt af to årsager. For det første fordi de anvendte ultralyds sensorer er nemmere at håndtere og billigere. For det andet er teknologien valgt fordi gruppen ikke ønskede at flyve højere end 5 meter. Skulle noget gå galt, er det langt farligere hvis dronen falder fra 20 end 5 meters højde.  
