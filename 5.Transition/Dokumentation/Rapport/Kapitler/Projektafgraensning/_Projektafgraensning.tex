%%%%%% Realisering af løsning %%%%%%
\chapter{Projektafgrænsning}
\label{chap:afgraensning}

Da Track'n'Play systemet skulle defineres var der et bredt udvalg af mulige funktionaliteter. Der blev blandt andet snakket om at implementere justering af lydens fase, så man oplevede ens lyd fra begge højtalere, Justering af lydstyrke, på baggrund af afstanden til anlægget samt visning af ønsket lydstyrke på et display. Vi overvejede også at koble anlægget på WiFi, så man derved kunne streame musik dertil fra sin smartphone, pc eller anden trådløs enhed.
Dog blev det fra projektets start besluttet at vi skulle holde fokus på de moduler der skulle udgøre system grundfunktionalitet. De ting som skulle skabe systemet grundfunktionalitet blev kaldt ``Need to have'', mens eventuelle forbedringer og udvidelser af systemt blev kaldt ``Nice to have'' tilføjelser. 

Need to have:
\begin{itemize}
\item Bestemmelse af position
\item Rotering af højtaler
\item Kabelforbindelse til medieafspiller
\item Komunikation mellem bæarbar enhed og anlægget
\item Måling af lydtryk i brugerens position
\item Lydforstærker
\item Interface til at vise og ændre ønsket lydstyrke
\end{itemize}
Nice to have:
\begin{itemize}
\item Justering af lydens fase
\item Trådløs medieafspilning
\end{itemize}
