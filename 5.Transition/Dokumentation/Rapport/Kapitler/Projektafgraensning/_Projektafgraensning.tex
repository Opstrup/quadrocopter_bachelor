\section{Projektafgrænsning}

Det er blevet vedtaget at implementere systemet som vist på figur ???????.
Der blev fra dag et af besluttet at afgrænse systemets omfang, således at der kunne sættes et realistisk mål i forhold til projekts omfang. \\
Et af afgrænsningerne var videokameraet. Det var tiltænkt at dronen skulle have et videokamera fast monteret, som skulle bruges til at live streame, i samme stil som hvis et stationært overvågningskamera blev anvendt.
Derudover blev der besluttet at nøjes med at bruge lavere kvalitets billeder, for  at gøre upload tiden mindre og derved fokusere på flyvningen.

I kravspecifikationen beskrives det at ethvert device med internet adgang kan benytte webapplikationen. Afhængigt af hvilket device der benytter webapplikationen, kræves en ny opsætning. Derfor er det besluttet kun at designe og implementere webapplikationen til en computer.

Flyvehøjde - bevidst valg af teknologi til højdemåling
Flyvehøjden til tests er bevidst valgt til at være under 5m. Dette skyldes valg af sensorer, da den valgte teknologi der anvendes har en maksimal rækkevidde på 5m. Derudover er højden valgt for at have mere kontrol over dronen under flyvning. 

Projektet blev opdelt i iterationer. Der blev valgt 4 iterationer, hvor den første iteration har den største prioritet, mens den sidste iteration prioriteres mindre højt. Dette medfører at man sikrer at de vigtigste dele af systemet implementeres først. Disse iterationer er beskrevet i projektets dokumentation[x].
