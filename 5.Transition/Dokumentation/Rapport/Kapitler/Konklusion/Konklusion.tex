\section{Konklusion}



Igennem projektforløbet er det lykkedes at implementere de essentielle moduler som skal bruges til håndtering af dronen. Det er derimod ikke lykkedes at samle disse moduler til en enhed der kan udføre autonom overvågning. 

Det er lykkedes at implementere en fungerende kommunikationslink mellem drone og server, derudover er det delvist lykkedes at lave styringen til dronen. Her virker modulerne enkeltvis, men systemet er ikke optimeret til at fungere som en samlet enhed. Dronens højdemåling, tilpasning af orientering og flyve funktionalitet virker, men fungerer langt fra optimalt.
Flyve orienteringen og flyve funktionaliteten er ikke blevet optimeret, da det har været svært at test flyve dronen. 


På webapplikationen er det lykkedes at implementere således at bruger kan logge på webapplikationen. Ved succesfuld login har bruger mulighed for at hente og oprette en ny flyveopsætning. Dog kan flyveopsætningen ikke sendes til serveren, da den kommunikationen ikke er blevet implementeret. Ved oprettelse af en ny flyveopsætning, bruges kortet til at finde de forskellige waypoints. Hver gang der trykkes på kortet oprettes et nyt waypoint, til dette waypoint kan der bestemmes om der skal tages et billede og i hvilken højde det skal tages.



Her gives en samlet konklusion på projektarbejdet. Hvad er lykkedes, hvad er evt. ikke 
lykkedes og årsagen til dette. Konklusionen skal gerne indeholde et klart budskab. 

Desuden sammenfattes de slutninger, der kan drages af de resultater, som er omtalt i rapportens tidligere afsnit. Konklusioner kan være såvel positive som negative. Man skal 
tage sig i agt for ikke at undertrykke de negative fund (hvis f.eks. en metode har vist sig 
uegnet, bør det opfattes som et bidrag til ens erfaringsmateriale, ikke som et personligt 
nederlag). 
I konklusionen trækkes desuden de store linier op. Væsentlige kvantitative resultater kan 
nævnes, hvorimod den detaljerede redegørelse og argumentationen henvises til diskussionen i rapportens hoveddel. 
Endelig kan der her gives en beskrivelse af forslagtil forbedringer af projektet/produktet