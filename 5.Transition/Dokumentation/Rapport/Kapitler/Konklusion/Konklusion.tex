\section{Konklusion}


I projektforløbet er det lykkedes at implementere en server, hovedparten af en webapplikation til opsætning af drone og en række enheder der tilsammen skulle udgøre dronen. Det er ikke lykkedes at færdigudvikle webapplikation og drone. Webapplikationen kan kun hente data fra server og enhederne der tilsammen skulle udgøre drone virker hver for sig, men ikke som samlet enhed.

Det er lykkedes at implementere kommunikationslink mellem drone og server, hvilket betyder dronen kan sende information om sin nuværende GPS position til server og hente flyveopsætninger fra server. Derudover kan dronen finde sin nuværende flyvehøjde, flyveretning og GPS position ved aflæsning af ultralydssensor, kompas og GPS. Ud fra det aflæste data tilpasser drone automatisk flyveindstillinger, men grundet dårlige vejrhold i implementeringsperioden og få testflyvninger kom automatisk ændring af flyveindstillinger ikke til at fungere optimalt.

Webapplikationen er udviklet så bruger ikke kan få lov at tilgå webapplikationens funktionalitet uden at være logget ind. Efter succesfuld login hentes og fremvises data fra server. 
Ved oprettelse af en ny flyveopsætning, bruges et kort til at markere waypoints. Hver gang der trykkes på kortet oprettes et nyt waypoint, og bruger bliver bedt om at indtaste hvorvidt der skal tages et billede på den valgte position og i hvilken højde billedet skal tages.
Når flyveopsætning er færdig udarbejdet trykker bruger \textit{upload}, men pga. manglede implementering kan webapplikationen ikke sende den nyoprettede flyveopsætning til server. 

