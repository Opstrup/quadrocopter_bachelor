\section{Konklusion}

Dette bachelorprojekt er udarbejdet for Ingeniørhøjskolen Aarhus Universitet. Målet for bachelorprojektet var at udvikle en overvågningsdrone, der ud fra brugers anvisninger autonomt kunne overvåge et defineret område. 

Projektets udviklingsforløb har været iterativ og agilt, og til projektstyring er der blevet gjort brug af RUP og ASE-modellen. RUP er brugt til at danne de overordnede rammer for projektet og til at sikre en klar sammenhænge mellem projektets faser og forskellige arbejdsopgaver. ASE modellen er brugt til at danne grundlag for løbende udarbejdelse af dokumentation. Til store dele af projektets systemarkitektur er N + 1 modellen anvendt.

I projektforløbet er det lykkedes at implementere en server, hovedparten af en webapplikation til opsætning af drone og en række enheder der tilsammen skulle udgøre dronen. Det er ikke lykkedes at færdigudvikle webapplikation og dronen. Webapplikationen kan kun hente data fra server og de enheder der tilsammen skulle udgøre drone virker hver for sig, men ikke som samlet enhed.

Det er lykkedes at implementere kommunikationslink mellem drone og server, hvilket betyder dronen kan sende information om sin nuværende GPS position til server og hente flyveopsætninger fra server. Derudover kan dronen finde sin nuværende flyvehøjde, flyveretning og GPS position ved aflæsning af ultralydssensor, kompas og GPS. Ud fra det aflæste data tilpasser drone automatisk flyveindstillinger, men grundet dårlige vejrhold i implementeringsperioden og få testflyvninger kom automatisk ændring af flyveindstillinger ikke til at fungere optimalt.

Webapplikationen er udviklet så bruger ikke kan få lov at tilgå webapplikationens funktionalitet uden at være logget ind. Efter succesfuld login hentes og fremvises data fra server. Ved oprettelse af en ny flyveopsætning, bruges et kort til at markere waypoints. Hver gang der trykkes på kortet oprettes et nyt waypoint, og bruger bliver bedt om at indtaste hvorvidt der skal tages et billede på den valgte position og i hvilken højde billedet skal tages.
Når flyveopsætning er færdig udarbejdet trykker bruger \textit{upload}, men pga. manglede implementering kan webapplikationen ikke sende den nyoprettede flyveopsætning til server. 

Da al kommunikation i systemet går til og fra server via et REST API, blev der i projektets udviklingsfase lagt stor fokus på at udvikle en velfungerende server. Det er lykkedes at implementere en fuld funktionel server, der fungerer som bindeled mellem drone og webapplikation. Serveren er implementeret som en passiv enhed, der er ansvarlig for håndtering af systemets data og tilgås via HTTP protokollen. Den udviklede server indeholder af en SQLite database. I SQLite databasen gemmes og hentes løbende information om systemets brugere, samt information om flyveopsætninger, flyveruter og billeder. 


På baggrund de opnåede resultater kan det konkluderes, at de grundlæggende principper bag systemet er eftervist og testet. Det er lykkedes at få  mange enheder og delsystemer til at fungerer efter hensigten, men det overordnede mål for projektet er ikke opfyldt på tilfredsstillende vis.