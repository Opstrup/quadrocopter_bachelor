%%%%%% Konklusion %%%%%%
\section{Fælles konklusion}
\label{chap:fallesKonklusion}

Målet for projektet var at udvikle et system, der automatisk kunne opdatere højtalerretning samt justere lydstyrke for at give bruger det bedst mulige lydbillede. %Et af delmålene var at udvikle et userinterface, hvorfra bruger nemt kunne ændre indstilling af ønsket lydstyrke.

Det lykkedes for gruppen af designe og implementere en fungerende lydtryksregulering, hvor lydtrykket blev målt og sendt trådløst til anlægget, der herefter kunne skrue op eller ned for højtalernes output. Integrationen af positionsbestemmelsen og højtalerplatformen blev også implementeret og fungerede fornuftigt, men krævede mere arbejde for at kunne opfylde samtlige krav fra kravspecifikationen. Konkluderende kan man sige at de grundlæggende principper bag systemet er blevet eftervist tilfredsstillende.


Specifikke elementer i tidsplanen skred lidt undervejs, da visse emner og problematikker var blevet undervurderet. Ved løbende at holde møder hvor tidsplan, deadlines og aftaler blev gennemgået og revurderet lykkes det at rette op på eventuelle fejl og mangler. Dermed lykkes det gruppen at blive færdig med projektet indenfor projektets overordnede tidsramme.

Gennemførelsen af dette projekt blev gjort ved brug af tværfaglig viden fra semesterets kurser.
Viden fra tidligere samt nuværende semester blev brugt igennem hele projektforløbet. Til projektstyring
blev RUP og V-modellen anvendt. Disse metoder og værktøjer blev brugt efter gruppens
emne og produktidé var fundet, og dannede baggrund for udarbejde af analyse og design.
V-modellen sikrede klar sammenhænge fra start til slut i projektet, så den røde tråd blev bevaret.
Mens RUP blev brugt til at håndhæve sammenhængen mellem de forskellige arbejdsopgaver og projektets overordnede faser. 

Dette projekt har været en lærerig proces for alle medlemmer af projektgruppen. Det lykkedes projektgruppens individer i fællesskab at håndtere de problemer og komplikationer der har været i løbet af projektet. En stor og svær opgave
blev i fællesskab løftet, projekts læringsmål er opfyldt og vi var alle tilfredse med projektforløbet og det opnåede slutresultat.  
Gruppen har erfaret, at de fleste problematikker i fælleskab kan overvindes. Samtidig har gruppen erfaret, at vi som individer er i stand til at sætte os ind i ukendt stof og bruge ny viden til at udvikle et system. Dette gør projektgruppen i stand til at udføre fremtidige opgaver. 

