\section{Resultater}

I dette afsnit beskriver resultatet af accepttesten som blev foretaget i projektets afsluttende fase. 
Accepttesten er bygget op om seks test cases, som bruges til at teste det samlede systems funktionalitet. For mere information om de forskellige test cases henvises til test dokumentet. [9] \\

Test case 1 er succesfuldt gennemført. Når dronen tændes, initieres main controller samt 3G/GPS modul og dronen opdaterer sin nuværende GPS position. Efter at have opdateret egen GPS position opretter dronen forbindelse til mobilt 3G-netværk og sender information om sin nuværende GPS position til server.

Test case 2 er delvist gennemført. Bruger har mulighed for at tilgå webapplikation, logge ind på applikationen og oprette nye flyveopsætninger. Webapplikationen kan både hente og vise information fra server. Det er dog ikke muligt at sende information fra webapplikation til server, da kommunikation mellem webapplikation og server ikke er fuldt ud implementeret. Dette betyder at bruger ikke kan uploade nyoprettede flyveopsætninger til server.  

Test case 3 er ligesom test case 2 delvist gennemført. Dronen kan sende sin nuværende GPS position til server og hente flyveopsætninger fra server. Det bemærkes dog, at flyveopsætninger, der hentes fra server, ikke er oprettet via webapplikationen men derimod tilføjet server via backend. 
Tilpasning af flyveindstillinger er implementeret i tre trin. Først tilpasses flyvehøjde, dernæst tilpasses flyveretning og når både flyvehøjde og flyveretning er tilpasset flyves fremad.
Tilpasning af flyveindstillinger er ikke blevet optimeret, hvilket får dronen til at afvige fra det ønskede resultat og gør dele af test case 3 fejler.

Test case 4, 5 og 6 er ikke godkendt, da funktionalitet kun er designet og ikke implementeret.
