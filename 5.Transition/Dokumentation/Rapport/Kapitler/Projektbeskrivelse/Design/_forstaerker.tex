%%%%%% Forstærker %%%%%%
\newpage
\subsection{Forstærker}

Forstærkeren består af de 3 trin indgangstrin, preamp og udgangstrin.
\begin{figure}[H]
  \centering
    \includegraphics[width=0.8\textwidth]{Billeder/Projektbeskrivelse/effekt_amp.png}
    \caption{Blok diagram over effektforstærkeren.}
    \label{fig:effekt_amp}
\end{figure}

\paragraph{Indgangstrin} Opgaven for indgangstrinnet er at filtrere uønskede signaler væk fra lydkilden. Det blev vurderet bedst at bruge aktive filtre, da det giver en fordel i forhold til belastningen af signalvejen, idet de overholder kædereglen. Valget faldt på sallen-key filtre, da disse er kendte fra undervisningen. Filteret består af to andenordens filtre, først et højpasfilter og dernæst et lavpasfilter. Rækkefølgen er valgt, da det giver en DC-blokering af indgangen og lavpasset fjerner støj fra højpasleddet.
Udregningerne af filtrene kan findes i Dokumentationsrapportens afsnit \vref{DokRap-subsec:effektForst}.

\paragraph{Preamp} Forstærkningen i preamp'en skulle, jævnfør kravspecifikationen, kunne ændres i et antal trin, for derved at ændre outputtet. Der blev fundet et digitalt potentiometer, som viste sig at kunne netop det systemet havde brug for. Potentiometeret havde to variable modstande, som blev digitalt styret af et \SI{6}{bit} mønster. Det digitale potentiometer blev anvendt til at justere forstærkningen i en operationsforstærker. Det digitale potentiometer skulle have \SI{6}{bit} parallelt. For at reducere pinforbruget på PSoC blev der anvendt skifteregistre med parallel udgang.

\paragraph{Udgangstrin} Trinnet er udregnet i forhold til en højtaler på \SI{4}{$\Omega$} og hvilke komponenter der var til rådighed. Det resulterede i en maksimal spidsstrøm på \SI{1,5}{A}, hvilket begrænser den effekt som trinnet kan levere. Desuden blev der designet en sikring af udgangen i tilfælde af, at den kortsluttes til stel.
Udregningerne af disse kan ligeledes findes i Dokumentationsrapportens afsnit \vref{DokRap-subsec:effektForst}.