%%%%%% Lydtryksmåling %%%%%%
\subsection{Lydtryksmåling}
Da lyd svinger meget i amplitude, skal det modtagne signal fra mikrofonen behandles. Dette er gjort ved at integrere signalet inden det samples af ADC'en på PSoC'en. Signalet fra mikrofonen bliver AC koblet, inden det forstærkes op af to sæt forstærkere, som henholdsvis forstærker \SI{101}{gg} og \SI{67}{gg}. Herefter integreres det kraftigt ved at bruge et passivt lavpas filter, som har en knækfrekvens på \SI{16}{Hz}. Sammen med en forstærkning over \SI{70}{dB} gør det, at signalet klipper, herved bruges længden af det klippet signal. Signalet analyseres på den bærbare enhed igennem et midlingsfilter, hvilket giver et udtryk for det pågældende lydtryk. På grund af systemets AC karakteristik reagerer det kraftigst på lave frekvenser som bas og stortromme, hvilket også er, hvor der er stor effektafsættelse fra lydgiveren.