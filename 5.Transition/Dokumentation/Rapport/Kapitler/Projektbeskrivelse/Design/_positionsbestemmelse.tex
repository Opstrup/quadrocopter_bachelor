%%%%%% Positionsbestemmelse %%%%%%
\subsection{Positionsbestemmelse}

Da vi midt i designfasen mistede en ultralydsmodtager, valgte vi et system med kun 2 modtagere. Dette medførte at vi ikke var i stand til at bestemme brugerens position, men i stedet kun vinkel mellem modtagerne og brugeren.\\
De benyttede ultralydsmodtagere havde indbygget båndpasfilter omkring \SI{40}{kHz}. Dette gjorde, at filtreringen af signalet skete i selve modtagerne, og det var unødvendigt med eksterne filtre inden analysering.\\
Systemet er opbygget på følgende måde.

\begin{figure}[H]
\centering
\includegraphics[width=0.9\textwidth]{Billeder/Projektbeskrivelse/Design/Positionsbestemmelse/position.pdf}\medskip
\caption{Opstilling for positionsbestemmelse}
\label{fig:opstillingPosBes}
\end{figure}



\subsubsection*{40 kHz sender}
Ultralydssenderen skal generere et \SI{40}{kHz} signal, som udsendes i bursts med en varighed på \SI{1}{ms} 10 gange i sekundet, jævnfør flowspecifikationen i Dokumentationsrapportens afsnit \vref{DokRap-subsec:flowSpec} og frem.

Figur \ref{fig:ultralydsSender} viser designet af ultralydssenderen. Den består af en signalgenerator, hvis formål er at generere det korrekte signal. Forstærkeren skal kunne levere nok energi til selve ultralydshøjtaleren, for ikke at belaste signalgeneratoren.

\begin{figure}[htb]
	\centering
	\includegraphics[width=0.90\textwidth]{Billeder/Projektbeskrivelse/Design/Positionsbestemmelse/ibd Sender.pdf}
	\caption{ibd, \SI{40}{kHz} ultralydssender}
	\label{fig:ultralydsSender}
\end{figure}

\paragraph{Signalgeneratoren} For ikke at trække veksler på PSoC'ens processorkraft designes signalgeneratoren ved brug af UDB-blokke\footnote{Universal Digital Block -- Programmérbar logik i PSoC'en.} samt hardware internt i PSoC'en, således den blot skal initialiseres for at kunne fungere. Figur \ref{fig:senderImplementeringHW} viser hvordan signalgeneratoren er implementeret.

Når PWM'en indstilles til en passende periode og dutycycle, vil NAND-gaten lade nyttesignalet passere i de ønskede bursts. Spændingsfølgeren Opamp\_1 er indført for ikke at belaste signalgeneratoren unødigt.

\begin{figure}[htb]
	\centering
	\includegraphics[width=0.90\textwidth]{Billeder/Projektbeskrivelse/Design/Positionsbestemmelse/Sender implementering.pdf} 
	\caption{Sender implementering i PSoC}
	\label{fig:senderImplementeringHW}
\end{figure}


\paragraph{Forstærker} For at designe forstærkeren skal der tages hensyn til slewrate, idet \SI{40}{kHz} har en relativ kort periode, kombineret med at forsyningen helst skal op på omkring \SI{20}{V}. Med disse værdier kræves der en mindste slewrate omkring \SI{1,6}{MV/s}.

Der skal desuden tages højde for gain bandwidth produktet idet forstærkeren som minimum skal forstærke \SI{4}{gg} ved \SI{40}{kHz}. Det mindste GBP kan herved udregnes til \SI{160}{kHz}.\newline
Forstærkeren skal forstærke inputtet på \SI{5}{V} nok til at signalet klippes til forsyningen på \SI{20}{V}. Derfor skal forstærkningen som minimum være \SI{4}{gg}. 

Den færdige forstærker designes til at forstærke \SI{-8,4}{gg}, have DC-arbejdspunkt ved half supply, samt have en båndbredde fra \SI{0,13}{kHz} til \SI{0,7}{MHz}


Det fulde design og implementering af ultralydssenderen, inklusiv beregninger, kan ses i afsnit \vref{DokRap-subsec:ultralydsSender} i Dokumentationsrapporten.


\subsubsection*{40 kHz mikrofon og forstærker}

I en afstand på \SI{2}{m} vil mikrofonen give et signal på ca. 10 til \SI{15}{mV}. Dette gør, at der skal bruges en forstærkning på over \SI{400}{gg}.\newline
Først blev det afprøvet at lave forstærkningen med én operationsforstærker, men dette viste sig ikke at kunne lade sig gøre, da båndbredden i forstærkeren, ved en frekvens på \SI{40}{kHz}, højst kunne forstærke ca. \SI{45}{gg}. Denne forstærkning var ikke nok til at modtage et ordenligt signal, hvorfor forstærkeren blev opdelt i to ens forstærkertrin med lavere forstærkning, for at deres grænsefrekvens lå over \SI{40}{kHz}.

Valget faldt på at lave to ikke-inverterende forstærkere for at holde kredsløbet så simpelt som muligt. For ikke at skulle bruge en +/- forsyning blev det valgt, at forstærkerne skulle have et DC-arbejdspunkt på \SI{2,5}{V}. For at undgå at forstærke DC-niveauet, blev der i tilbagekoblingen indsat en kondensator.

\subsubsection*{40 kHz detektor}
Detektoren omdanner \SI{40}{kHz} signalet fra forstærkeren til et logisk \SI{5}{V} signal. Når mikrofonen opfanger et signal, går udgangen fra denne blok høj.
Der bliver benyttet en spændingsmultiplikator til at omdanne AC-spændingen fra forstærkeren til en DC-spænding med samme amplitude. Denne DC-spænding detekteres af en komparator, som trigger ved \SI{700}{mV}.

\subsubsection*{Tidsmåling på PSoC}
For at bestemme vinkelen til bærbar enhed skal tidsforskellen mellem de to \SI{40}{kHz} detektorer bestemmes. Jo større vinkel desto større vil tidsforskellen være. Figur \ref{fig:PSoC_Topview} viser hvordan tidsmålingen er implementeret i PSoC'en.

XOR-gaten fungerer ved at udgangen kun er høj når netop ét input er højt. Når vinkelen er forskellig fra $0\gr$ i forhold til centerposition, vil den ene detektor modtage ultralyden før den anden, se eventuelt figur \ref{fig:opstillingPosBes} ovenfor. Dette udnyttes idet timeren kun aktiveres når netop én detektor modtager signalet.\newline
Når den anden detektor modtager signalet vil XOR-gaten's udgang gå lav, hvilket dels vil medføre at timeren stopper sin tælling, dels vil aktivere interruptet isr\_Read. Interruptrutinen aflæser blot timerens nuværende værdi og resetter den.

\begin{figure}[htb]
	\centering
	\includegraphics[width=0.9\textwidth]{Billeder/Projektbeskrivelse/Design/Positionsbestemmelse/PSoC topview.png}
	\caption{Implementering af tidsmåling}
	\label{fig:PSoC_Topview}
\end{figure}

Ulempen ved det nuværende design er, at det ikke er muligt at skelne mellem hvilken detektor, der først modtager signalet. Med andre ord kan implementeringen kun se til den ene side i forhold til centerpositionen.\newline
Desuden vil der komme en ekstra aktivering af timeren, når ultralydsburstet har passeret, idet dette først vil ske for den første detektor, efterfulgt af den anden detektor. Dette fænomen er der dog set bort fra i implementeringen idet det antages, at den ekstra aktivering vil resultere i samme tidsmåling som den første, og derfor ikke har nogen indvirkning på målingerne.