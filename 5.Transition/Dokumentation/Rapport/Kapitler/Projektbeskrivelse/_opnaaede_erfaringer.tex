%%%%%% Opnåede erfaringer %%%%%%
\section{Opnåede erfaringer}
\label{chap:opnaaede_erfaringer}

I løbet af projektets forløb er der opnået mange nye erfaringer, hvoraf nogle vil blive gennemgået i følgende afsnit.

Under projektets indledende faser opnåede gruppen viden om projektstrukturering og projektstyring. Der blev arbejdet i henhold til RUP og V-modellen, hvilket virkede rigtigt godt. De to arbejdsmodeller gav både et bedre flow i projektarbejdet og sikrede, at arbejdsopgaver ikke blev glemt. 

Der lå en stor udfordring i at danne en syvmandsgruppe og det kunne til tider være svært at overskue projektet. For at skabe overblik, blev der lavet tidsplaner og afholdt korte møder, hvor status blev opsummeret. På denne måde kunne eventuelle problemstillinger blive behandlet tidligt.\newline
Gruppen har desuden været god til at holde diskussioner konstruktive og tage en fælles beslutning inden diskussionerne eskalerede.


Store dele af projektet har ikke været direkte bearbejdet i undervisningen i semesterets kurser, hvorfor det har været nødvendigt at indsamle megen ny viden på egen hånd.


\subsection{Nye arbejdsredskaber}
Gruppen valgte at anvende SVN, da dette værktøj er godt til fildeling samt versionshistorik. Vi har tidligere benyttet MS Word til dokumentation- og rapportskrivning. Det blev dog besluttet hurtigt at vi havde brug for et andet værktøj til dette. Gruppen valgte derfor at tage et kursus i tekstredigeringsprogrammet LaTeX, der udmærker sig ved at sikre en ensartet udformning af dokumenter. programmet har en stejl indlæringskurve, men gruppens medlemmer har været gode til at hjælpe hinanden.