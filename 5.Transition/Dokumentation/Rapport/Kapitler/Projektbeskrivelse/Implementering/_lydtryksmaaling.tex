%%%%%% Lydtryksmåling %%%%%%
\subsection{Lydtryksmåling}
Lydtryksmåleren blev realiseret på fumlebræt og funktionaliteten blev testet. Herefter blev kredsløbet realiseret på et veroboard og gentestet. Undervejs i projektforløbet blev designet ændret, hvilket medførte, at lydtryksmåleren på veroboardet blev kasseret og det nye design blev lavet på fumle bræt. Opstillingen kan ses på figur \ref{fig:Fumle_opstilling}.


På figur \ref{fig:lydtryk} ses et oscilloskop-billede af udgangen på lydtryksmåleren. På billedet ses det uintegrerede signal for neden henholdsvis integrerede signal for oven. Det ses, at et højt lydtryk giver et bredt og klippet signal.

\begin{figure}[H]
  \centering
    \includegraphics[width=0.6\textwidth]{Billeder/Projektbeskrivelse/lydtryk.jpg}
    \caption{Test af lydtryksmåler}
    \label{fig:lydtryk}
\end{figure}

Da systemet har en kraftig integration af signalet med et lavpas filter, har det en opladningstid før systemet er pålideligt. 
Det tager \SI{10}{ms} fra systemet tændes, til systemet er klart og det er sikkert at sample på signalet.