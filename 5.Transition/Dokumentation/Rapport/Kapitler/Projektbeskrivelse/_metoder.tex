%%%%%% Metoder %%%%%%
\section{Metoder}
\label{chap:metoder}

Projektet er gennemført ved hjælp af forskellige arbejdsmetoder. Disse er anvendt for at skabe overblik over og sammenhæng i projektet. Følgende arbejdsmetoder er blevet anvendt:

\begin{itemize}
	\item RUP (Rational Unified Process)
	\item V-modellen
\end{itemize}

\subsubsection*{RUP -- Rational Unified Process}
RUP blev anvendt under hele projektet og var med til at skabe en god struktur og gav et bedre flow.
Faserne i RUP giver nogle gode deadlines, som gruppen fulgte. Dette gjorde at alle medlemmer af gruppen hele tiden var på samme stadie i projektet. De forskellige faser blev inddelt, nogle hvor hele projektgruppen arbejdede sammen og andre hvor projektgruppen blev delt i mindre grupper.
\begin{figure}[H]
\centering
\includegraphics[width=1.0\textwidth]{Billeder/Projektbeskrivelse/RUPSummaryDiag}
\caption{RUP faserne og de opgaver/discipliner de indebærer}
\label{fig:RUP_diagram}
\end{figure}




\subsubsection*{V-model}
Figur \ref{fig:V_model} viser hvordan V-modellen er blevet anvendt i projektforløbet.

Under inceptionfasen defineres kravene til systemet og samtidig specificeres tests til at eftervise, at kravene er opfyldt. Testene samles i en accepttest, inden arkitekturdesign påbegynde, og udføres som en afsluttende test.

\begin{figure}[htb]
	\centering
	\includegraphics[width=0.9\textwidth]{Billeder/Projektbeskrivelse/Vmodel}
	\caption{V-model anvendt i projektet}
	\label{fig:V_model}
\end{figure}

Enhedstest udføres under implementering, for løbende at kontrollere hvorvidt de anvendte teknologier kan anvendes. Dette gøres, fordi flere teknologier er ukendte, og designet derved kan revideres og ændres iterativt uden komplikationer. Enheder testes op imod de specificerede krav samt flowspecificeringer, for efterfølgende at sættes sammen og testes i integrationstesten.