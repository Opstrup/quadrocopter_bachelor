%%%%%% Test %%%%%%
\section{Test}

\label{chap:test}

Indenfor hver del af projektet er der udført tests for at garantere funktionaliteten, men også til at sørge for at samlingen, af de enkelte moduler, bliver så gnidningsfri som mulig. Ved at lave gode tests opfanger man eventuelle fejl og mangler inden integrationen af projektets moduler. Det kan i sidste ende betyde at man undgår et eventuelt tidsspild på at rette fejl sent.


I projektet er der udført følgende forskellige typer tests:


\textbf{Enheds-/modultest} \\
Her bliver hver enhed/modul testet for sig selv. Der testes på den basale funktionalitet og eventuelle krav fra kravspecifikationen. Der testes blandt andet om App’en udsender de rigtige frekvenser.


\textbf{Integrationstest} \\
Her bliver systemets dele sat sammen del for del og herefter testet. For eksempel testes der om de udsendte frekvenser fra App’en kan detekteres igennem guitaren på Effektpedalen (PSoC’en).


\textbf{Accepttest} \\
I denne del testes det endelige produkt i forhold til accepttestspecifikationen. Dette er en sluttest som oftest foretages med kunden (vejleder) tilstede, således begge parter er enige om resultatet.
For yderligere og konkret information om dokumenterede tests og resultater, henvises til dokumentationsrapportens kapitel 6 på side 111.