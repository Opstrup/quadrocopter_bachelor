%%%%%% Test %%%%%%
\section{Test}

\label{chap:test}

I projektforløbet er der løbende blevet udført test. Indledningsvis udføres enhedstest i takt med nye moduler/enheder udvikles, senere laves integrationstest der tester kommunikation og samarbejde mellem flere enheder og til sidst udføres accepttest.

Nedenfor ses en kort beskrivelse af de udførte tests:\\


\textbf{Enhedstest} \\
Enhedstest udføres løbende i takt med nye enheder udvikles. Testene udarbejdes for at sikre kvalitet, funktionalitet og grænseflader af de nyudviklede enheder. 

Hovedformålet med enhedstest er at teste tidligt i projektforløbet for at opdage eventuelle fejl og mangler. Hvilket i sidste ende kan spares meget tid og besvær, da fejl og mangler ofte er svære og mere tidskrævende at rette sent i et projektforløb.


\textbf{Integrationstest} \\
I integrationstest testes koblingen mellem to eller flere enheder. Her testet om grænseflader mellem enhederne fungerer og om enhederne kan kommunikere og arbejde sammen.  



\textbf{Accepttest} \\
Accepttesten er en todelt test der tester systemet som helhed. 
Først udføres accepttest af funktionelle krav og dernæst testes ikke-funktionelle krav.
Accepttest af funktionelle krav foregår via en gennemgang af use cases, som bruges til at kontrollere systemets funktionalitet. Ikke-funktionelle krav bruges til at teste systemspecifikationer.  

For yderligere og konkret information om de forskellige tests og tilhørende resultater henvises til testdokumentet[X] i dokumentationen.