%%%%%% Lydtryksmåling %%%%%%
\newpage
\subsection{Højtalerplatforme}
Til at udforme højtalerplatformene skulle der bruges et par højtalere og to motorer. 

Højtalerne blev valgt ud fra kriterier som størrelse, mobilitet og vægt. Det blev besluttet, at der skulle bruges nogle små computerhøjtalere, idet de var lette, og dermed ville mindske belastningen på motorerne. 

Motorerne blev valgt ud fra, hvor meget de kunne trække, hvordan de roterede og deres interface. Gruppen var på udkig efter en stærk motor, der som minimum kunne rotere $180\gr$ og havde et simpelt interface. De tre mest oplagte motorer til højtalerplatformene var DC-, stepper- og servomotorer.\newline
DC-motorer var interessante, fordi de kunne rotere i begge retninger og var nemme at starte og stoppe ved indkobling henholdsvis frakobling af forsyning.\newline
Steppermotorer var præcise, kunne køre i begge retninger, og kunne køre i både fullstep og halfstep, men krævede til gengæld en smule opsætning og programmering.\newline
Servomotorer mindede meget om DC-motorerne, dog krævede de, ud over forsyning, også et kontrolsignal. Fordelen var til gengæld at servomotorerne havde indbygget feedback, hvilket gjorde dem simplere at arbejde med. Derfor blev det besluttet, at der skulle bruges servomotorer til højtalerplatformene.
