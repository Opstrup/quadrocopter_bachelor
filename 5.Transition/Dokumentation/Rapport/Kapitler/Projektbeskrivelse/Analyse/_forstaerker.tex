%%%%%% Forstærker %%%%%%
\subsection{Forstærker}

For at have fuld kontrol over filtrering, gain og effekt, er det valgt at bygge en forstærker fra bunden. Der ønskes filtrering af lydsignalet, så kun hørbar lyd bliver afspillet i højtaleren. Gain’et skal styres gennem en aktuator, som et udtryk for et ønsket lydtryksniveau. Brugeren skal ikke have direkte adgang til gain’et. Forstærkeren skal kunne drive \SI{4}{$\Omega$}- og \SI{8}{$\Omega$}-højtalere.
Brugeren skal desuden kunne tilslutte en medieafspiller.

Der var flere overvejelser omkring hvilken klasse udgangstrinnet skulle tilhøre. Valget lå mellem klasse A, B og AB.\newline
A blev fravalgt grundet strømforbrug og B blev fravalgt grundet crossoverforvrængning. Klasse AB blev valgt, fordi det reducerer de negative sider ved rene klasse A og B forstærkere.