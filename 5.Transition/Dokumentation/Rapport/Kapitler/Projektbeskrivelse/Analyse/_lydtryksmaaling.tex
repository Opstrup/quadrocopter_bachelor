%%%%%% Lydtryksmåling %%%%%%
\subsection{Lydtryksmåling}

En af de mest centrale dele i systemet er, at tilpasse lydtryk hos brugeren. For at kunne tilpasse lydtrykket ved brugeren er det nødvendigt at kunne måle det. 

Følgende tre teknologier til måling af lydtrykket blev undersøgt: Teoretisk beregning af lydtrykket på baggrund af afstand mellem lydkilde og bruger, brug af piezo-elektrisk element og brug af mikrofon. 

Når afstanden mellem lydkilden og brugeren var kendt, kunne ønskede lydtryk beregnes, da lydtrykket er omvendt proportional med afstanden. Ulempen ved denne metode var, at det ville give en høj kobling mellem lydtryksmåleren og positionsbestemmelsen, og den høje kobling vil gøre systemet mindre hårdfør overfor fejlramte enheder.

Det piezo-elektriske element blev fravalgt som teknologi, da de piezo-elektriske elementer gruppen havde til rådighed, ikke var følsomme nok over for lydtryk i luften.  Elementerne egnede sig bedre til måling af vibrationer i et fast materiale.

Måling af lydtrykket med mikrofon simpelt og nemt at fungere. Samtidig gav denne teknologi en lav kobling i systemet, hvilket gjorde systemet mere robust over for fejl i andre delsystemer. Derfor blev en mikrofon valgt til den teknologi der skulle bruges til måling af lydtrykket.    