%%%%%% Krav %%%%%%
\section{Krav}

For at definere systemet er der opstillet en række krav. Aktører til systemet er blevet identificeret og use cases er opstillet for at definere de funktionelle krav. Desuden er der opstillet ikke-funktionelle krav til systemet som helhed og til systemets blokke.\newline

En mere detaljeret beskrivelse af kravene er uddybet i projektets kravspecifikation[x]

\subsection{Funktionelle krav}
Der er defineret 2 aktører i systemet: Bruger \& GPS-satellit.
Bruger er en primær aktør der ønsker at initialisere systemet. Det er bruger der tilslutter de nødvendige moduler til dronen og opretter flyveopsætningen på webapplikationen.
GPS-satellit er en sekundær aktør der giver dronen mulighed for at finde dens egen position. 

På billedet vises aktør diagrammet for systemet. Diagrammet viser hvad aktørerne har fat i eller bruger.
\begin{figure}[H]
	\centering
	\includegraphics[width=0.60\textwidth]{Billeder/Krav/Use_case_diagram}
	\caption{Use case diagram}
	\label{fig:useCaseDiagram}
\end{figure}

Alle use cases beskrevet i projektets kravspecifikation er specificeret som fully dressed use cases, hvor hovedforløb er successcenariet når use casen er gennemført. \\
Nedenfor vises en beskrivelse af de forskellige use cases og en beskrivelse af deres hovedforløb:

\textbf{Use case 1: Start drone} \\
Bruger tilslutter batteri og dronen initialiseres. Dronen sender dens nuværende position til webapplikationen.\\

\textbf{Use case 2: Ny flyveopsætning} \\
Bruger logger på webapplikation og opretter en ny flyveopsætning. Flyveopsætningen sættes tilgængelig for dronen.\\

\textbf{Use case 3: Flyv til position}\\
Dronen henter flyve opsætningen fra serveren og påbegynder flyvningen. Dronen indstiller sig efter den ønskede flyvehøjde og flyver mod ønsket position. \\

\textbf{Use case 4: Billede af position} \\
Dronen ankommer til den ønskede position og tager et billede. Hvis billedet accepteres, flyver dronen videre mod næste koordinat. \\

\textbf{Use case 5: Vis tidligere flyvninger} \\
Bruger tilgår webapplikationen for at se tidligere flyvninger.\\

\textbf{Use case 6: Anti kollision} \\
Dronens anti kollisions sensorer detekterer et objekt og et undvigningsmanøvre udføres. \\

\newpage
\subsection{Ikke-funktionelle krav}




De ikke-funktionelle krav specificerer krav til blandt andet timing, afstande og dimensioner samt lydniveauer. 

I Dokumentationsrapportens afsnit \vref{DokRap-sec:ikkeFunkKrav} er alle de ikke-funktionelle krav opskrevet.

Kravene er grupperet efter hvor i systemet de hører til. De generelle krav er de krav, som ikke kan placeres under et bestemt modul af systemet. Eksempelvis er der opstillet krav til kommunikationen mellem bærbar enhed og det resterende system.

Der er opstillet krav til højtalere og forstærker, hvor blandt andet default indstillingen, der nævnes i ovenstående use case, defineres.\newline
Desuden er tider, som for eksempel den maksimale tid det må tage for højtalerne at rotere for fuldt udslag, defineret.

Den bærbare enhed er en vigtig komponent i systemet, og der er opstillet krav til denne for at sikre at den giver så få gener for en bruger som muligt. Blandt andet er der sat krav til batterilevetiden.

For at sikre integriteten af systemet er der opstillet krav til opsamlingen af data, således der for eksempel ikke kan måles og reageres på en ugyldig lydtryksmåling, eller hvis den sidste værdi er for gammel. Dette er gjort for at hindre pludselige ændringer af lyden under afspilning, eller når musikken stopper, at der ikke skrues fuldt op.


