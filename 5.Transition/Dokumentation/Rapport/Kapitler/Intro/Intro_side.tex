\chapter{Intro}

\section{Revisionshistorik}
\begin{table}[H]
	\centering
		\begin{tabular}{|p{1.7 cm}|p{2 cm}|p{2.5 cm}|p{6.8 cm}|} 
		\hline
			\textbf{Rev. Nr} & \textbf{Dato}		& \textbf{Initialer} 	& \textbf{Ændring} \\ \hline
			1.0 	& 1-12-2014 & AO, KG, RL  & Oprettet dokument.  \\ \hline
			1.1 	& 1-12-2014 & RL  & Tilføjet resume, indledning, forord og systemskitse. \\ \hline
			1.2 	& 1-12-2014 & KG  & Tilføjet krav og projektafgrænsning. 	\\ \hline
			1.3 	& 1-12-2014 & AO  & Tilføjet arbejdsmetode og \newline udviklingsproces.	\\ \hline

		\end{tabular}
	\caption{Revisionshistorik}
	%\label{tab:TC1}
\end{table}


\vspace{1.5cm}

\section{Ordforklaring}
\begin{table}[H]
	\centering
		\begin{tabular}{|p{2.6cm}|p{4.5 cm}|p{6.5 cm}|} 
		\hline
			\textbf{Forkortelse} & \textbf{Betydning} & \textbf{Forklaring} \\ \hline
			 Drone & Drone & Aeroquad ARF quadrocopter og \newline påmonteret hardware. \\ \hline
		\end{tabular}
	\caption{Ordforklaring}
	%\label{tab:TC1}
\end{table}

\vspace{1cm}

\section{Indledning}
I dette kapitel beskrives systemarkitektur og design. Målet med kapitlet er at beskrive og designe de blokke systemet består af. Dels gives der indsigt i hvordan blokkene kommunikerer med hinanden og hvordan den interne og eksterne kommunikation virker. Desuden vises hvordan systemets blokke er designet. 