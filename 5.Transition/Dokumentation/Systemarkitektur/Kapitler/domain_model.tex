\section{Domain Model}

Domain modellen bruges som en overgang mellem kravspecifikation og systemarkitektur. 
I kravspecifikation beskrives hvad der sker ved interaktion med systemet. Mens systemarkitekturen bruges til at beskrive systemet i blokke og til at skitsere de interne og eksterne forbindelser. Domain modellen bruges til at beskrive hele systemets domæne. Der kigges ikke på hardware vs. software, der kigges i stedet på "enheder" og deres ansvarsområder.

På figur \ref{fig:domain_model} vises domain model tilhørende systemet. De fire øverste enheder i domain modellen dækker ansvarsområder som har med systemets webapplikation at gøre. De resterende enheder er alle tilknyttet ansvarsområder der omhandler dronen.

På domain modellen vises at bruger kommunikerer med dronen gennem en website. Websiden har forbindelse til en server som videresender instruktioner til main controller. Det vises desuden at kommunikation mellem server og main controller går gennem det mobile 3G netværk. Under flyvning styrer main controlleren både kamera, GPS, afstandssensorer og drone. 

\vspace{-5pt}
%kommentar
\begin{figure}[H]
	\centering
	\includegraphics[width=1.\textwidth]{Billeder/domain_model.png}
	\caption{Domain model}
	\label{fig:domain_model}
\end{figure}

\newpage

Som det fremgik af figur \ref{fig:domain_model} er domain modellen opbygget af tre forskellige typer klasser. Nedenfor beskrives de tre forskellige klasser:

\textbf{Domain klasser}\newline
Domain klasser repræsenterer systemets domæne. Domain klasser er passive elementer, der er ansvarlige for væsentlige delparter af systemets funktionalitet.  

\textbf{Boundary klasser}\newline
Boundary klasser repræsenterer use case-aktører og aktørernes grænseflader til systemet. Boundary klasser er elementer der ligger i den yderste periferi af systemet. I nogle tilfælde er boundary klasser "front-end" elementer, der er designet til at håndtere output samt at tage imod input fra systemets bruger. I andre tilfælde er boundary klasser "back-end" elementer der bruges til at hjælpe/understøtte controller klasser.

\textbf{Controller klasser} \newline
Controller klasser indeholder systemets domænelogik og står for at kontrollere systemets boundary og domain klasser. Desuden skal controller klasserne også håndtere input og output fra systemets boundary klasser.

\vspace{1cm}

Domain modellen bruges ikke i beskrivelsen af arkitekturen af systemet hardware. Men i afsnittet hvor systemets software arkitektur beskrives gøres stor brug af domain modellen. I software afsnittet bruges domain modellens klasser til at danne grundlag for systemets softwareklasser, bla. bruges klasse navne og klassernes indbyrdes forhold.






