\chapter{Systemarkitektur}

\section{Revisionshistorik}
\begin{table}[H]
	\centering
		\begin{tabular}{|p{2 cm}|p{2 cm}|p{3 cm}|p{6 cm}|} 
		\hline
			\textbf{Rev. Nr} & \textbf{Dato}		& \textbf{Initialer} 	& \textbf{Ændring} \\ \hline
			1.0 	& & &  \\ \hline
			1.1 	& & &	\\ \hline
		\end{tabular}
	\caption{Revisionshistorik}
	%\label{tab:TC1}
\end{table}

\vspace{1.5cm}

\section{Ordforklaring}
\begin{table}[H]
	\centering
		\begin{tabular}{|p{2.5cm}|p{4.5 cm}|p{6.5 cm}|} 
		\hline
			\textbf{Forkortelse} & \textbf{Betydning} & \textbf{Forklaring} \\ \hline
			 bdd& Block definition diagram  & Viser systemet blokke  \\ \hline
			 ibd& Internal block diagram & Viser ekstern og intern kommunikation tilhørende en blok \\ \hline
		\end{tabular}
	\caption{Ordforklaring}
	%\label{tab:TC1}
\end{table}

\vspace{2cm}

\section{Indledning}
I dette kapitel vil systemarkitekturen blive beskrevet. Målet med kapitlet er at beskrive de blokke systemet består af. Dels gives der indsigt i hvordan blokkene kommunikerer med hinanden, desuden gives der beskrivelser af den interne kommunikation i systemets blokke. 