\chapter{Intro}

\section{Revisionshistorik}
\begin{table}[H]
	\centering
		\begin{tabular}{|p{1.7 cm}|p{2 cm}|p{2.5 cm}|p{6.8 cm}|} 
		\hline
			\textbf{Rev. Nr} & \textbf{Dato}		& \textbf{Initialer} 	& \textbf{Ændring} \\ \hline
			1.0 	& 11-09-2014 & AO, KG, RL  & Oprettet dokument, og lavet udkast til hardware diagrammer.  \\ \hline
			1.1 	& 16-09-2014 & KG, RL  & Opdateret block definition og internal block diagrammer. 	\\ \hline
			1.2 	& 22-09-2014 & KG  & Tilføjet signal beskrivelse til alle hardware diagrammer. 	\\ \hline
			
			2.1 	& 29-09-2014 & AO, KG, RL  & Oprettet software afsnit. Påbegyndt logical view ved at tilføje designoverview og pakkediagrammer.	\\ \hline
			2.2 	& 10-10-2014 & AO, KG, RL  & Tilføjet sekvens- og klassediagrammer tilhørende første iteration.	\\ \hline
			2.3 	& 31-10-2014 & AO, KG, RL  & Tilføjet sekvens- og klassediagrammer tilhørende anden iteration.	\\ \hline
			2.4 	& 11-11-2014 & AO, KG, RL  & Tilføjet sekvens- og klassediagrammer tilhørende tredje og fjerde iteration.	\\ \hline
			2.5 	& 13-11-2014 & RL  & Tilføjet state machines til alle iterationer.	\\ \hline
			2.6 	& 14-11-2014 & AO  & Tilføjet data view.	\\ \hline
			2.7 	& 17-11-2014 & AO, RL  		& Tilføjet proces view.	\\ \hline
			2.8 	& 25-11-2014 & AO, KG, RL  & Tilføjet deployment view.	\\ \hline
			2.9 	& 28-11-2014 & AO, KG, RL  & Tilføjet implementation view.	\\ \hline
		\end{tabular}
	\caption{Revisionshistorik}
	%\label{tab:TC1}
\end{table}


\vspace{1.5cm}

\section{Ordforklaring}
\begin{table}[H]
	\centering
		\begin{tabular}{|p{2.6cm}|p{4.5 cm}|p{6.5 cm}|} 
		\hline
			\textbf{Forkortelse} & \textbf{Betydning} & \textbf{Forklaring} \\ \hline
			 Drone & Drone & Aeroquad ARF quadrocopter og \newline påmonteret hardware. \\ \hline
			 Webapp & Webapplikation & Applikation som bruges til opsætning af nye flyvninger og monitorering af tidlige flyvninger. \\ \hline
			 Server & Webserver & Server dækker over systemet database og tilhørende logik. \\ \hline
			 Main controller & Main controller  & Main controller er en arduino mega2560. Main controlleren styrer dronen ud fra flyveopsætning og data fra diverse sensorer.   \\ \hline
			 Flight control \newline board & Flight control board  & Boardet er købt sammen med quadrocopter hos aeroquad. Boardet bruges som mellemled mellem main controller og dronens motorer.  \\ \hline
			 3G/GPS & 3G/GPS modul  & Dette modul er et 3G/GPS shield der påmonteres main controller  \\ \hline			 
			 
			 bdd& Block definition diagram  & bdd bruges til at vise systemets overordnede hardware blokke  \\ \hline
			 ibd& Internal block diagram & ibd bruges til at vise ekstern og intern kommunikation tilhørende en eller flere blokke. \\ \hline
			 sd& Sekvens diagram & Diagrammet bruges til at vise afvikling og timing af forskellige processer. \\ \hline
			 stm& State machine diagram & stm viser states i en given proces. \\ \hline
		\end{tabular}
	\caption{Ordforklaring}
	%\label{tab:TC1}
\end{table}

\vspace{3.5cm}

\section{Indledning}
I dette kapitel beskrives systemarkitektur og design, der er udarbejdet på baggrund af kravspecifikationen. Målet med kapitlet er at beskrive og designe de blokke systemet består af. Dels gives der indsigt i hvordan blokkene kommunikerer med hinanden og hvordan den interne og eksterne kommunikation virker. Desuden vises hvordan systemets blokke er designet. 