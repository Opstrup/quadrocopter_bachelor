\chapter{Systemarkitektur}

\section{Revisionshistorik}
\begin{table}[H]
	\centering
		\begin{tabular}{|p{2 cm}|p{2 cm}|p{3 cm}|p{6 cm}|} 
		\hline
			\textbf{Rev. Nr} & \textbf{Dato}		& \textbf{Initialer} 	& \textbf{Ændring} \\ \hline
			1.0 	& & &  \\ \hline
			1.1 	& & &	\\ \hline
		\end{tabular}
	\caption{Revisionshistorik}
	%\label{tab:TC1}
\end{table}

\vspace{1.5cm}

\section{Ordforklaring}
\begin{table}[H]
	\centering
		\begin{tabular}{|p{2.5cm}|p{4.5 cm}|p{6.5 cm}|} 
		\hline
			\textbf{Forkortelse} & \textbf{Betydning} & \textbf{Forklaring} \\ \hline
			 &  &  \\ \hline
			 &  & \\ \hline
		\end{tabular}
	\caption{Ordforklaring}
	%\label{tab:TC1}
\end{table}

\vspace{2cm}

\section{Indledning}
Dette kapitel beskriver hvilke enheder systemet består af samt grænseflader mellem
enhederne. Til at beskrive hardware og tilhørende grænseflader benyttes SysML diagrammer, mens software og tilhørende grænseflader beskrives med UML diagrammer.