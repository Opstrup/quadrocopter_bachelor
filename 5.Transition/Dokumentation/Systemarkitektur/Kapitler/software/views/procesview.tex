\section{Process view}

Ingen af systemets centrale dele gør brug af tråde. Men systemet er opbygget således at afvikling af drone, website og server kan betragtes som 3 samtidige processer. Alle 3 dele bruges sideløbende i runtime og imellem dem flyder løbende data.

For systemets bruger ser det ud som om al information går direkte fra website til drone eller omvendt. Men reelt set er websitet og drone aldrig i direkte kontakt med hinanden. I stedet går al kommunikation til og fra serveren, som fungerer som et kommunikationslag. 

\vspace{-5pt}
\begin{figure}[H]
	\centering
	\includegraphics[width=0.7\textwidth]{Billeder/process_view}
	\vspace{0cm}
	\caption{Process view}
	\label{fig:process_view}
\end{figure}

\textbf{Server}\\
Serveren er passiv, hvilket betyder den aldrig tager initiativ til udveksling af data. Serveren foretager ingenting på egen hånd, den står i stedet og venter på at blive igangsat af drone eller website. 
 
\textbf{Drone} \\
I drone processen håndteres al styring af drone og kommunikation mellem drone og server. Når dronen er tændt og 3G/GPS modulet initialiseret, opdaterer dronen med få minutters interval egen GPS position for efterfølgende at sende serveren information om positionen. Desuden kontrollerer dronen løbende om der er en ny flyveopsætning tilgængelig på serveren. Hvis der er en ny flyveopsætning tilgængelig hentes den, og en ny flyvning påbegyndes. 

\textbf{Website}\\
Mellem server og website er der lavet en socket connection. Det betyder at indholdet af websitet opdateres hver gang der kommer nyt indhold på serveren. Desuden bruges websitet når systemets bruger ønsker at lave nye flyveopsætninger eller opdatere de nuværende.\\


Der er lagt megen energi i at designe og bygge systemet på en vis der tillader udvidelser og tilføjelser. Dels kan server nemt tilgås af flere forskellige websites og desuden kan websitet nemt håndtere flere droner og client’er samtidig.

\newpage
\subsection{Kommunikations timing}
For at de forskellige enheder i systemet kan kommunikere optimalt, er der blevet designet nogle bestemte kommunikations timing tabeller som skal overholders for at kunne kommunikere optimalt.\\
Disse tabeller skal køres i følgende rækkefølge:

\begin{enumerate}
	\item Dronen tændes
	\item Handlingen i tabel et udføres og gentages hvert 5min
	\item Handlingen i tabel to udføres og gentages hvert 5min
	\item Handlingen i tabel to retunere andet end 0
	\item Handlingen i tabel tre udføres
	\item Dronen er nu klar til at flyve
	\item Handlingen i tabel to gentages ikke mere
	\item Efter flyvning gentages det hele, bortset fra step 1
\end{enumerate}

\begin{table}[H]
	\centering
		\begin{tabular}{| m {1cm} | m {3cm} | m {4cm} | m {7cm} |}
			\hline
			Enhed & HTTP Request & Data & Handling \\ \hline
			Drone & PUT & Online og Location & Opdatere sin position og sætter sig online\\ \hline
			Server & Response & 200 OK & Giver ok svar tilbage \\ \hline
		\end{tabular}
	\caption{Kommunikation imellem drone og server step 1}
	\label{tab:kom_drone_server_1}
\end{table}

\begin{table}[H]
	\centering
		\begin{tabular}{| m {1cm} | m {3cm} | m {4cm} | m {7cm} |}
			\hline
			Enhed & HTTP Request & Data & Handling \\ \hline
			Drone & GET & NextEvent & Efterspørg nextevent værdig \\ \hline
			Server & Response & NextEvent & Retunere drone data, med nextevent \\ \hline
		\end{tabular}
	\caption{Kommunikation imellem drone og server step 2}
	\label{tab:kom_drone_server_2}
\end{table}

\begin{table}[H]
	\centering
		\begin{tabular}{| m {1cm} | m {3cm} | m {4cm} | m {7cm} |}
			\hline
			Enhed & HTTP Request & Data & Handling \\ \hline
			Drone & GET & Waypoint & Dronen henter waypoints tilhørende event\\ \hline
			Server & Response & Waypoints & Retunere waypoints tilhørende event \\ \hline
			Drone & PUT & NextEvent = 0 & Dronen ændre sin NextEvent værdig til 0 \\ \hline
			Server & Response & 200 OK & Giver ok svar tilbage \\ \hline
		\end{tabular}
	\caption{Kommunikation imellem drone og server step 3}
	\label{tab:kom_drone_server_3}
\end{table}