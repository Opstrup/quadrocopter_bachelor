\section{Deployment view}

I dette afsnit beskrives hvilke softwareklasser der bruges i systemets mest grundlæggende moduler. Desuden beskrives hvilke protokoller der anvendes mellem de grundlæggende moduler i systemet, fx. beskrives layout af meddelelser der sendes mellem drone og server.


\begin{figure}[H]
\centering
\includegraphics[width=1\textwidth]{Billeder/deployment_overview.png}
\caption{Overordnet deployment diagram}
\label{fig:deployment_generel}
\end{figure}


På illustrationen figur \ref{fig:systemskitse_deployment} vises systemet opdelt i de 3 mest grundlæggende moduler. Som det ses af illustrationen består systemet overordnet set af drone, webapplikation og sever. I de følgende afsnit vil de tre overordnede moduler blive pakket op og forklaret hver for sig, og desuden vil HTTP 1.1 protokollen blive beskrevet.



\newpage
\subsection{Drone}
På denne side beskrives hvordan drone modulet er opbygget.
Dronen foruden stel og motorer består af  

\begin{figure}[H]
\centering
\includegraphics[width=1\textwidth]{Billeder/deployment_drone.png}
\caption{Systemskitse}
\label{fig:deployment_drone}
\end{figure}

\newpage
\subsection{Server}

\begin{figure}[H]
\centering
\includegraphics[width=1\textwidth]{Billeder/deployment_server.png}
\caption{Systemskitse}
\label{fig:deployment_server}
\end{figure}

\newpage
\subsection{Webapplikation}

\begin{figure}[H]
\centering
\includegraphics[width=1\textwidth]{Billeder/deployment_webapp.png}
\caption{Systemskitse}
\label{fig:deployment_webapp}
\end{figure}

\newpage
\subsection{HTTP protokol}
HTTP protokollen er valgt som kommunikations led. 

Derudover understøttes HTTP protokollen også af 3G/GPS modulet, hvilket er den primære årsag til at HTTP protokollen blev valgt som kommunikations lag.
HTTP er en af de mest anvendte protokoller når der kommer til kommunikation fra webapplikation til server. 

 
HTTP protokol version 1.1 er valgt fremfor 1.0, da version 1.1 indeholder metoden PUT, som er en vigtig metode for systemet.

Metoderne der anvendes er GET, PUT og POST. Når en af disse metoder anvendes, kommer der et svar retur, som indeholder en række informationer.
Svaret indeholder en header og en body, som vist på Figur \ref{fig:headerbodyget}. 

Headeren indeholder informationer om HTTP forbindelsen. Den første linje indeholder en status besked, som viser hvorvidt beskeden er blevet modtaget korrekt eller om der er sket en fejl. Ydermere inderholder headeren tidspunktet for modtagelsen, hvilket format body'en er i, tilladte metoder til kommunikation og hvilken type server der er hentet fra. Selve body'en starter på figur \ref{fig:headerbodyget} efter 6b og slutter ved 0, hvor 6b er antal karakterer i body'en vist i hexadecimal. Når PUT eller POST anvendes, er det nødvendigt at sende information for headeren med. Derudover er det vigtigt at sende antal karakterer i bodyen med. Ved PUT og POST modtages der også et svar i form af en header og en body, med de samme informationer som ved GET metoden. 
\begin{figure}[H]
	\centering
	\includegraphics[width=1\textwidth]{Billeder/header_body_get.png}
	\caption{Header og Body for GET metoden}
	\label{fig:headerbodyget}
\end{figure}
