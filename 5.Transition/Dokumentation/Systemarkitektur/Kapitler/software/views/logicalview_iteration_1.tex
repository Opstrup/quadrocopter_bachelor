\subsection{Iteration \#1}
I iteration 1 arbejdes der med blokke som dækker over systemets mest grundlæggende funktionalitet. Batteri,
ESC’er, motorer og sensorer tilsluttes dronen. Server opsætning så kommunikation imellem dronen og websitet kan føre sted, her henvises der til data view'et for flere detaljer vedrørende serveren. Desuden gøres dronen i stand til oprette forbindelse til websitet via 3G-shield’et. Hvordan systemet er tiltænkt at bruges beskrives i user story nedenfor:

\subsubsection*{User story}
Bruger tænder dronen ved at tilslutte batteri. Main controller samt 3G/GPS module initialiseres og nuværende GPS position opdateres. Herefter oprettes forbindelse mellem drone og websitet, og information om dronen er online samt information om dronens nuværende GPS position sendes til websitet. Fra websitet er det muligt for bruger løbende at observere hvorvidt dronen er online og på hvilken GPS position dronen sidst har befundet sig.

%kommentar
\begin{figure}[H]
	\centering
	\includegraphics[width=1\textwidth]{Billeder/design_overview/design_overview_iteration1.png}
	\vspace{-.5cm}
	\caption{Design overview \#iteration 1}
	\label{fig:design_overview_UC1}
\end{figure}


\newpage
\subsubsection*{Sekvens diagram - drone}
På sekvensdiagrammet på figur \ref{fig:Sekvens_diagram_iteration1}, vises hvilke klasser der indgår og bruges i første iteration. Af sekvensdiagrammet fremgår det, at sekvensen først startes når bruger tilslutter batteri og tænder dronen. Når der er tilkoblet forsyning initialiseres main controller samt 3G/GPS og nuværende GPS position  (longitude og latitude) opdateres. Dronens nuværende GPS position opdateres når dronen sender PUT requests til websitet. PUT requests bruges dels til at fortælle websitet at dronen er online og dels til at give websitet information om dronens nuværende position. 


%kommentar
\begin{figure}[H]
	\centering
	\includegraphics[width=0.93\textwidth]{Billeder/sekvens/sekvens_iteration1}
	\caption{Sekvens diagram \#iteration 1}
	\label{fig:Sekvens_diagram_iteration1}
\end{figure}
\newpage

På figur \ref{fig:Sekvens_diagram_initialget} og figur \ref{fig:Sekvens_diagram_putDroneStatus} bliver 3G modulet uddybet yderligere ifht. iteration 1.
\ref{fig:Sekvens_diagram_initialget} viser hvilke klasser der anvendes til at hente værdier fra serveren. De data der hendes ned med GET requesten, skal bruges sammen med PUT request i den anden klasse. For at kunne udføre et PUT funktion, skal de id'er der ønskes at blive sendt stemme overens med dem der ligger på serveren.

\begin{figure}[H]
	\centering
	\includegraphics[width=0.93\textwidth]{Billeder/sekvens/sekvens_iteration1_initialget}
	\caption{Sekvens diagram \#iteration 1 - initialget}
	\label{fig:Sekvens_diagram_initialget}
\end{figure}

\begin{figure}[H]
	\centering
	\includegraphics[width=0.93\textwidth]{Billeder/sekvens/sekvens_iteration1_putdronestatus}
	\caption{Sekvens diagram \#iteration 1 - putDroneStatus}
	\label{fig:Sekvens_diagram_putDroneStatus}
\end{figure}

\subsubsection*{Sekvens diagram - website}
På figur \ref{fig:Sekvens_diagram_login} ses sekvens diagrammet over login på websitet. På diagrammet vises det at useren først bliver ført til næste side ved succesfuld login. Diagrammet viser også kommunikationen med CRUDServiceDrone som kommunikerer med databasen. Diagrammet viser også hvordan AuthenticationServices setter den givet users data inden useren bliver ført vider i systemet.
\begin{figure}[H]
	\centering
	\includegraphics[width=0.93\textwidth]{Billeder/sekvens/login_sq_diagram.png}
	\caption{Sekvens diagram login}
	\label{fig:Sekvens_diagram_login}
\end{figure}
\newpage




\subsubsection*{State machine diagram}
\vspace{-0.1cm}
I state machine diagrammet på figur \ref{fig:Statemachine_iteration1}, vises de forskellige states der eksisterer i iteration 1 og hvordan flowet imellem dem ser ud. Der eksisterer givet vis kun 2 states i iteration 1, men state machinen er medtaget fordi den på nem og overskuelig vis illustrerer systemflowet.
%kommentar
\begin{figure}[H]
	\centering
	\includegraphics[width=1\textwidth]{Billeder/statemachine/State_iteration1.png}
	\vspace{-0.5cm}
	\caption{Statemachine \#iteration 1}
	\label{fig:Statemachine_iteration1}
\end{figure}
\newpage

\subsubsection*{Klasse diagram}
\vspace{-0.1cm}
Figur \ref{fig:classDiagram_iteration1} vises et klassediagram tilhørende iteration 1. Klassediagrammet viser iterationens vigtigste klasser, samt deres tilhørende metoder og attributter. På den følgende side forefindes en kort beskrivelse klasserne og deres metoder.

%kommentar
\begin{figure}[H]
	\centering
	\includegraphics[width=1\textwidth]{Billeder/klasse_diagrammer/classdiagram_iteration1.png}
	\vspace{-0.5cm}
	\caption{Klassediagram \#iteration 1}
	\label{fig:classDiagram_iteration1}
\end{figure}

\textbf{Main.cpp} \\
Main.cpp filen bruges til at sætte arduino board korrekt op, bla. sættes baudrate på de forskellige serielle forbindelser. Desuden bruges Main.cpp til at kalde og eksekverer forskellige klasse, objekter og funktioner.

\textbf{GPS} \\
GPS klassen fungerer som en virtuel klasse, og sikre som minimum implementering af init og updateGPSPosition uanset hvilken GPS der bruges. Klassen er lavet fordi der i udgangspunkt var mulighed for at bruge 3 forskellige slags GPS med 3G/GPS shieldet. 

\textbf{StandaloneGPS}\\
Denne klasse er ansvarlig for al kommunikation med GPS'en når standalone mode er valgt. 

\newpage

Nedenfor ses figur \ref{fig:udvidet3G_it1} som er en udvidet klasse diagram over 3G modulet. Klasse diagrammet viser hvilke metoder der indgår i iteration 1 for 3G modulet. 

\begin{figure}[H]
	\centering
	\includegraphics[width=0.50\textwidth]{Billeder/klasse_diagrammer/udvidet3G_iteration1.png}
	\vspace{0.5cm}
	\caption{Udvidet klasse diagram for 3G modulet - iteration 1}
	\label{fig:udvidet3G_it1}
\end{figure}


\textbf{GetAndPut} \\
GetAndPut klassen er den klasse der er tættest på hardwaren. Det er denne klasse der håndterer selve PUT og GET requesten på selve modulet. 
Denne metode henter data ned fra en server og getmetoden sorterer efterfølgende det modtagede data. 

\textbf{Communication} \\
Communication klassen håndterer sammen med andre klasser, alt der har med 3G at gøre. 
I klassen bruges to forskellige slags http request. Når der ønskes at trække information fra sever til drone gøres der brug af GET request mens der gøres brug af PUT request når dronen ønsker at informere server om ny lokation.

\textbf{EventHandler} \\
EventHandleren er den klasse der håndterer Events. EventHandleren er bindeledet mellem communication- og GetAndPut klassen. 





\newpage

\subsubsection*{Klasse diagram - website}
\vspace{-0.1cm}
Figur \ref{fig:classDiagram_login} viser klassediagrammet tilhørende websitet for iteration 1. De enkle klasser er beskrevet mere i detaljer her under.
\begin{figure}[H]
	\centering
	\includegraphics[width=1\textwidth]{Billeder/klasse_diagrammer/login_class_diagram.png}
	\vspace{-0.5cm}
	\caption{Klassediagram login}
	\label{fig:classDiagram_login}
\end{figure}

\newpage

\textbf{login} \\
Login view filen er det generede html dokument som useren ser på sin webclient. Denne fil er two-way databinded med loginCtrl klassen og via denne binding kan der sendes og modtages data direkte.

\textbf{loginCtrl} \\
LoginCtrl er controller klassen som er forbundet med view klassen via two-way databinding. Klassen uddelegere også opgaver til dens services.

\textbf{AuthenticationServices} \\
AuthenticationServices er den service der afgøre om useren er authenticated eller ej. Hvis useren er authenticated bliver han redirected til websitets home-page. Klassen gør brug af UserServices til at gemme information omkring den nuværende user der er logget ind i systemet.

\textbf{UserServices}\\
UserServices indeholder information omkring den nuværende user der er logget ind i systemet.

\textbf{CRUDServiceDrone} \\
CRUDServiceDrone er den service der styre alt kontakt til databasen i systemet. Igennem denne service er det muligt at hente, opdater og poste data til databasen.